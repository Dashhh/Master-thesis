%
% Niniejszy plik stanowi przykład formatowania pracy magisterskiej na
% Wydziale MIM UW.  Szkielet użytych poleceń można wykorzystywać do
% woli, np. formatujac wlasna prace.
%
% Zawartosc merytoryczna stanowi oryginalnosiagniecie
% naukowosciowe Marcina Wolinskiego.  Wszelkie prawa zastrzeżone.
%
% Copyright (c) 2001 by Marcin Woliński <M.Wolinski@gust.org.pl>
% Poprawki spowodowane zmianami przepisów - Marcin Szczuka, 1.10.2004
% Poprawki spowodowane zmianami przepisow i ujednolicenie 
% - Seweryn Karłowicz, 05.05.2006
% Dodanie wielu autorów i tłumaczenia na angielski - Kuba Pochrybniak, 29.11.2016

% dodaj opcję [licencjacka] dla pracy licencjackiej
% dodaj opcję [en] dla wersji angielskiej (mogą być obie: [licencjacka,en])
\documentclass[en]{pracamgr}

% Dane magistranta:
\autor{Adam Starak}{361021}

% TODO[Dodaj tytuł]
\title{Title in English}
\titlepl{Tytuł po polsku}

%\tytulang{An implementation of a difference blabalizer based on the theory of $\sigma$ -- $\rho$ phetors}

%kierunek: 
% - matematyka, informacyka, ...
% - Mathematics, Computer Science, ...
\kierunek{Computer Science}

% informatyka - nie okreslamy zakresu (opcja zakomentowana)
% matematyka - zakres moze pozostac nieokreslony,
% a jesli ma byc okreslony dla pracy mgr,
% to przyjmuje jedna z wartosci:
% {metod matematycznych w finansach}
% {metod matematycznych w ubezpieczeniach}
% {matematyki stosowanej}
% {nauczania matematyki}
% Dla pracy licencjackiej mamy natomiast
% mozliwosc wpisania takiej wartosci zakresu:
% {Jednoczesnych Studiow Ekonomiczno--Matematycznych}

% \zakres{Tu wpisac, jesli trzeba, jedna z opcji podanych wyzej}

% Praca wykonana pod kierunkiem:
% (podać tytuł/stopień imię i nazwisko opiekuna
% Instytut
% ew. Wydział ew. Uczelnia (jeżeli nie MIM UW))
\opiekun{dr Michał Pilipczuk\\
  Instytut Informatyki\\
  }

% miesiąc i~rok:
\date{May 2017}

%Podać dziedzinę wg klasyfikacji Socrates-Erasmus:
\dziedzina{ 
%11.0 Matematyka, Informatyka:\\ 
%11.1 Matematyka\\ 
%11.2 Statystyka\\ 
11.3 Informatyka\\ 
%11.4 Sztuczna inteligencja\\ 
%11.5 Nauki aktuarialne\\
%11.9 Inne nauki matematyczne i informatyczne
}

%Klasyfikacja tematyczna wedlug AMS (matematyka) lub ACM (informatyka)
%TODO - dodać klasyfikację
\klasyfikacja{D. Software\\
  D.127. Blabalgorithms\\
  D.127.6. Numerical blabalysis}

%TODO - dodać słowa kluczowe]
% Słowa kluczowe:
\keywords{parameterized algorithm}

% Tu jest dobre miejsce na Twoje własne makra i~środowiska:

\newtheorem{defi}{Definition}
\newtheorem{theorem}{Theorem}
\newtheorem{lemma}{Lemma}
\newtheorem{claim}{Claim}

\usepackage{chngcntr}
\counterwithin{theorem}{chapter}
\counterwithin{defi}{chapter}
\counterwithin{lemma}{chapter}

% koniec definicji

\begin{document}
\maketitle

%tu idzie streszczenie na strone poczatkowa
%TODO - dodaj abstract
\begin{abstract}
  W~pracy przedstawiono prototypową implementację blabalizatora
  różnicowego bazującą na teorii fetorów $\sigma$-$\rho$ profesora
  Fifaka.  Wykorzystanie teorii Fifaka daje wreszcie możliwość
  efektywnego wykonania blabalizy numerycznej.  Fakt ten stanowi
  przełom technologiczny, którego konsekwencje trudno z~góry
  przewidzieć.
\end{abstract}

\tableofcontents
%\listoffigures
%\listoftables

\chapter*{Introduction}
\addcontentsline{toc}{chapter}{Introduction}

Blabalizator różnicowy jest podstawowym narzędziem blabalii
fetorycznej.  Dlatego naukowcy z~całego świata prześcigają się
w~próbach efektywnej implementacji.  Opracowana przez prof. Fifaka
teoria fetorów $\sigma$-$\rho$ otwiera w~tej dziedzinie nowe
możliwości.  Wykorzystujemy je w~niniejszej pracy.

\chapter{Basic definitions}\label{r:pojecia}

\section{Structures}

\begin{defi}\label{Graph}
 	Graph
\end{defi}

%TODO - Tree? Forest?

\begin{defi}\label{Star}
	Star
\end{defi}

\begin{defi}\label{Spaning tree}
	Spanning tree
\end{defi}
+Additional notation: e.g. $deg_G(v)$


\section{Parameterized complexity}

\begin{defi}\label{Parameterized problem}
	Parameterized problem
\end{defi}

\begin{defi}\label{FPT algorithm}
	FPT algorithm
\end{defi}

\begin{defi}\label{Kernel}
	Kernel
\end{defi}

\begin{defi}\label{Kernelization algorithm}
	Kernelization algorithm
\end{defi}

\section{Graph decomposition}

\begin{defi}\label{Pathwidth}
	Path decomposition and pathwidth
\end{defi}

\begin{defi}\label{Treewidth}
	Tree decomposition and treewidth
\end{defi}

\begin{defi}\label{nice tree decomposition}
	Nice tree decomposition
\end{defi}

\chapter{Spanning Star Forest Problem}\label{r:losers}

For a given graph $G$, we say that $G'$ is a \emph{Spanning Star Forest} $S$
if every connected component $C$ is a star. In the 
\emph{Spanning Star Forest Problem} given a graph $G$, the objective is
to determine whether there exists a \emph{Spanning Star Forest}. It turns
out that the problem formulated in such a way is relatively simple. Although, 
various parametrizations described in this paper make it more complex.

\begin{lemma}\label{SSF lemma}
 A graph $G$ has a Spanning Star Forest if and only if it does not contain
 any isolated vertices.
\end{lemma}

\begin{theorem}
	Decision version of Spanning Star Forest Problem can be solved in linear
	time.
\end{theorem}

\section{Obtaining a solution}

In this section the focus will be set on obtaining an arbitrary solution for
a given instance of the \emph{Spanning Star Forest Problem}.

\begin{theorem}
	A solution for a Spanning Star Forest Problem can be found in linear time.
\end{theorem}

\section{Spanning Star Forest parameterized by the number of stars}

In the \emph{Spanning Star Forest Problem} parameterized by the number of
stars, given a graph $G$ and a natural number $k$, the objective is to
determine whether there exists a \emph{Spanning Star Forest} $S$ such that
the number of components is less than $k$.

It is natural to ask whether one can find a solution that minimizes the number
of connected components. Even though the problem looks slightly different
than the previous one, \emph{Spanning Star Forest} parameterized by the 
number of stars is NP-Complete. The following theorem proves the statement:

\begin{theorem}
	Spanning Star Forest Parameterized by the number of stars is NP-Complete.
\end{theorem}

\begin{lemma}
	There exists a reduction from Spanning Star Forest parameterized by the
	number of stars to Dominating Set.
\end{lemma}


\begin{thebibliography}{99}
\addcontentsline{toc}{chapter}{Bibliografia}

\bibitem[Bea65]{beaman} Juliusz Beaman, \textit{Morbidity of the Jolly
    function}, Mathematica Absurdica, 117 (1965) 338--9.

\bibitem[Blar16]{eb1} Elizjusz Blarbarucki, \textit{O pewnych
    aspektach pewnych aspektów}, Astrolog Polski, Zeszyt 16, Warszawa
  1916.

\bibitem[Fif00]{ffgg} Filigran Fifak, Gizbert Gryzogrzechotalski,
  \textit{O blabalii fetorycznej}, Materiały Konferencji Euroblabal
  2000.

\bibitem[Fif01]{ff-sr} Filigran Fifak, \textit{O fetorach
    $\sigma$-$\rho$}, Acta Fetorica, 2001.

\bibitem[Głomb04]{grglo} Gryzybór Głombaski, \textit{Parazytonikacja
    blabiczna fetorów --- nowa teoria wszystkiego}, Warszawa 1904.

\bibitem[Hopp96]{hopp} Claude Hopper, \textit{On some $\Pi$-hedral
    surfaces in quasi-quasi space}, Omnius University Press, 1996.

\bibitem[Leuk00]{leuk} Lechoslav Leukocyt, \textit{Oval mappings ab ovo},
  Materiały Białostockiej Konferencji Hodowców Drobiu, 2000.

\bibitem[Rozk93]{JR} Josip A.~Rozkosza, \textit{O pewnych własnościach
    pewnych funkcji}, Północnopomorski Dziennik Matematyczny 63491
  (1993).

\bibitem[Spy59]{spyrpt} Mrowclaw Spyrpt, \textit{A matrix is a matrix
    is a matrix}, Mat. Zburp., 91 (1959) 28--35.

\bibitem[Sri64]{srinis} Rajagopalachari Sriniswamiramanathan,
  \textit{Some expansions on the Flausgloten Theorem on locally
    congested lutches}, J. Math.  Soc., North Bombay, 13 (1964) 72--6.

\bibitem[Whi25]{russell} Alfred N. Whitehead, Bertrand Russell,
  \textit{Principia Mathematica}, Cambridge University Press, 1925.

\bibitem[Zen69]{heu} Zenon Zenon, \textit{Użyteczne heurystyki
    w~blabalizie}, Młody Technik, nr~11, 1969.

\end{thebibliography}

\end{document}


%%% Local Variables:
%%% mode: latex
%%% TeX-master: t
%%% coding: latin-2
%%% End:
