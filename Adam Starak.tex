%
% Niniejszy plik stanowi przykład formatowania pracy magisterskiej na
% Wydziale MIM UW.  Szkielet użytych poleceń można wykorzystywać do
% woli, np. formatujac wlasna prace.
%
% Zawartosc merytoryczna stanowi oryginalnosiagniecie
% naukowosciowe Marcina Wolinskiego.  Wszelkie prawa zastrzeżone.
%
% Copyright (c) 2001 by Marcin Woliński <M.Wolinski@gust.org.pl>
% Poprawki spowodowane zmianami przepisów - Marcin Szczuka, 1.10.2004
% Poprawki spowodowane zmianami przepisow i ujednolicenie 
% - Seweryn Karłowicz, 05.05.2006
% Dodanie wielu autorów i tłumaczenia na angielski - Kuba Pochrybniak, 29.11.2016

% dodaj opcję [licencjacka] dla pracy licencjackiej
% dodaj opcję [en] dla wersji angielskiej (mogą być obie: [licencjacka,en])
\documentclass[en]{pracamgr}

% Dane magistranta:
\autor{Adam Starak}{361021}

% TODO[Dodaj tytuł]
\title{Title in English}
\titlepl{Tytuł po polsku}

%\tytulang{An implementation of a difference blabalizer based on the theory of $\sigma$ -- $\rho$ phetors}

%kierunek: 
% - matematyka, informacyka, ...
% - Mathematics, Computer Science, ...
\kierunek{Computer Science}

% informatyka - nie okreslamy zakresu (opcja zakomentowana)
% matematyka - zakres moze pozostac nieokreslony,
% a jesli ma byc okreslony dla pracy mgr,
% to przyjmuje jedna z wartosci:
% {metod matematycznych w finansach}
% {metod matematycznych w ubezpieczeniach}
% {matematyki stosowanej}
% {nauczania matematyki}
% Dla pracy licencjackiej mamy natomiast
% mozliwosc wpisania takiej wartosci zakresu:
% {Jednoczesnych Studiow Ekonomiczno--Matematycznych}

% \zakres{Tu wpisac, jesli trzeba, jedna z opcji podanych wyzej}

% Praca wykonana pod kierunkiem:
% (podać tytuł/stopień imię i nazwisko opiekuna
% Instytut
% ew. Wydział ew. Uczelnia (jeżeli nie MIM UW))
\opiekun{dr Michał Pilipczuk\\
  Instytut Informatyki\\
  }

% miesiąc i~rok:
\date{May 2017}

%Podać dziedzinę wg klasyfikacji Socrates-Erasmus:
\dziedzina{ 
%11.0 Matematyka, Informatyka:\\ 
%11.1 Matematyka\\ 
%11.2 Statystyka\\ 
11.3 Informatyka\\ 
%11.4 Sztuczna inteligencja\\ 
%11.5 Nauki aktuarialne\\
%11.9 Inne nauki matematyczne i informatyczne
}

%Klasyfikacja tematyczna wedlug AMS (matematyka) lub ACM (informatyka)
%TODO - dodać klasyfikację
\klasyfikacja{D. Software\\
  D.127. Blabalgorithms\\
  D.127.6. Numerical blabalysis}

%TODO - dodać słowa kluczowe]
% Słowa kluczowe:
\keywords{parameterized algorithm}

% Tu jest dobre miejsce na Twoje własne makra i~środowiska:

\usepackage{chngcntr}
\usepackage{amsthm}
\usepackage{amsmath}
\usepackage[]{algorithm2e}
\usepackage{enumitem}

\newtheorem{defi}{Definition}
\newtheorem{theorem}{Theorem}
\newtheorem{lemma}{Lemma}
\newtheorem{claim}{Claim}
\newtheorem{corollary}{Corollary}
\newcommand{\ssf}{\emph{Spanning Star Forest}}
\newcommand{\ssfp}{\emph{Spanning Star Forest Problem}}
\newcommand{\domset}{\emph{Dominating Set}}
\newcommand{\domsetp}{\emph{Dominating Set Problem}}
\newcommand{\kssf}{\emph{Spanning Star Forest Problem} parameterized by the number of stars}
\newcommand{\cnfsat}{\emph{CNF-SAT}}
\newcommand{\ssfe}{\emph{Spanning Star Forest Extension Problem}}
\newcommand{\tsat}{\emph{3-SAT}}

\counterwithin{theorem}{chapter}
\counterwithin{defi}{chapter}
\counterwithin{lemma}{chapter}
\counterwithin{corollary}{chapter}
\counterwithin{claim}{chapter}

% koniec definicji

\begin{document}
\maketitle

%tu idzie streszczenie na strone poczatkowa
%TODO - dodaj abstract
\begin{abstract}
  W~pracy przedstawiono prototypową implementację blabalizatora
  różnicowego bazującą na teorii fetorów $\sigma$-$\rho$ profesora
  Fifaka.  Wykorzystanie teorii Fifaka daje wreszcie możliwość
  efektywnego wykonania blabalizy numerycznej.  Fakt ten stanowi
  przełom technologiczny, którego konsekwencje trudno z~góry
  przewidzieć.
\end{abstract}

\tableofcontents
%\listoffigures
%\listoftables

\chapter*{Introduction}
\addcontentsline{toc}{chapter}{Introduction}

Blabalizator różnicowy jest podstawowym narzędziem blabalii
fetorycznej.  Dlatego naukowcy z~całego świata prześcigają się
w~próbach efektywnej implementacji.  Opracowana przez prof. Fifaka
teoria fetorów $\sigma$-$\rho$ otwiera w~tej dziedzinie nowe
możliwości.  Wykorzystujemy je w~niniejszej pracy.

\chapter{Basic definitions}\label{r:pojecia}

\section{Structures}

A simple graph $G$ is a pair $(V,E)$ where $V$ denotes a set of vertices and $E$ denotes a set of undirected edges. Let $deg_G(v)$ denote a degree of vertex v in graph $G$. Let $G \setminus \{v\}$ be the abbreviation for $G'=(V(G) \setminus \{v\}, E(G) \setminus \{(u,v): u \in V(G)\})$. For a set $X \subset V(G)$ we denote $G[X]$ as a graph induced by vertices from $X$. A \emph{tree} $T$ is a graph where two vertices are connected by excatly one path. A \emph{spanning tree} $T$ of a graph $G$ is a graph which includes all of the vertices of $G$, with minimum possible number of edges. A \emph{star} $S$ is a tree of size at least 2 for which at most 1 vertex has a degree greater than $1$. A vertex in a \emph{star} that has the greatest degree is called a \emph{center} while the others are called \emph{rays}.

All the further definitions are taken from 
%TODO: cytat do parameterized algorithms%

\section{Parameterized complexity}

\begin{defi}\label{Parameterized problem}
	A parameterized problem is a language $L \subseteq \ \Sigma^* \times \mathbf{N}$, where $\Sigma$ is a fixed, finite alphabet. For an instance $(x,k) \in L$, $k$ is called the parameter.
\end{defi}

\begin{defi}\label{FPT algorithm}
	For a parameterized problem $Q$, an FPT algorithm is an algorithm $\mathcal{A}$ which, for any input $(x,k)$, decides whether $(x,k) \in Q$ in time $\mathcal{O}(f(k)\cdot n^c)$ where c is independent of $n,c$ and $f$ is a computable function.
\end{defi}

\begin{defi}\label{Kernel}
	A kernel for a parameterized problem $Q$ is an algorithm $\mathcal{A}$ that, given an instance $(x,k) \in Q$, works in polynomial time and returns an equivalent instance $(x',k') \in Q$
	such that $|x'| + k' \leq g(k)$ for a computable function $g$.
\end{defi}

\section{Graph decomposition}

\subsection{Path decomposition}

A path decomposition of a graph $G$ is a sequence $\mathcal{P} = (X_1, X_2, ..., X_r)$ of bags where $X_i \subseteq V(G)$ for each $i \in \{1,2, ..., r\}$ such that the following equations hold:
\begin{itemize}
	\item[(P1)] $\bigcup^r_{i=1} X_i = V(G)$.
	\item[(P2)] For every edge $(v,u) \in E(G)$ there exists a bag $X_p$ such that $u,v \in X_p$.
	\item[(P3)] For every $u \in V(G)$, if $u \in X_i \cap X_j$ for some $i < k$, then $u \in X_k$ for every $i < k < j$.
\end{itemize}
The width of a decomposition $\mathcal{P}$ is denoted as $w(P) = \max_{1\leq i\leq r} |X_i| - 1$. The pathwidth of a graph $G$ is the minimum possible width of a path decomposition i.e. $pw(G) = \min_{\mathcal{P}} w(P)$.\\

\noindent
A nice path decomposition of a graph $G$ is a path decomposition $\mathcal{P}$ that satisfies:
\begin{itemize}
	\item $X_0 = \emptyset$
	\item For every $i \in \{1,2,...,r\}$ there is either a vertex $v \notin X_i$ such that $X_{i+1} = X_i \cup \{v\}$ or there is a vertex $w \in X_i$ such that $X_{i+1} = X_i \setminus \{w\}$.
\end{itemize}

Nice path decomposition form is useful for designing dynamic-programming algorithm. It is worth to mention that any path decomposition $\mathcal{P}$ can be transformed into a nice path decomposition $\mathcal{P'}$ is polynomial time.

\subsection{Tree decomposition}

A tree decomposition is a generalization of a path decomposition. Formally, a tree decomposition of a graph $G$ is a pair $\mathcal{T} = (T, \{X_t\}_{t\in V(T)})$ where $\mathcal{T}$ is a tree whose every node $t$ is assigned to a vertex subset $X_t \subseteq V(G)$, called a bag, such that the following three conditions hold:
\begin{itemize}
	\item[(T1)] $\bigcup_{t\in V(T)}X_t = V(G)$.
	\item[(T2)] For every $(v,u) \in E(G)$ there exists a bag $t$ of $\mathcal{T}$ such that $v,u \in X_t$.
	\item[(T3)] For every $v \in V(G)$ the set $T_v = \{t \in V(T): v \in X_t\}$ induces a connected subtree of T.
\end{itemize}

Similarly, the width of a tree decomposition $\mathcal{T} = (T,\{X_t\}_{t\in V(T)})$, denoted as $w(\mathcal{T})$, is equal to $\max_{t \in V(T)} |X_t| - 1$. The treewidth of a graph $G$, denoted as $pw(G)$, is the minimal width over all tree decompositions of $G$.
\\
\\
A nice tree decomposition of a graph $G$ is a tree decomposition $(T, \{X_t\}_{t \in V(T)})$ such that
\begin{itemize}
	\item $X_i = \emptyset$ is $i$ is either root or leaf.
	\item Every non-leaf node is of one of the three following types:
	\begin{itemize}
		\item \textbf{Introduce node}: a node $t$ with exactly one child $t'$ such that $X_t = X_{t'} \cup \{v\}$ for some vertex $v \notin X_{t'}$.
		\item \textbf{Forget node}: a node $t$ with exactly one child $t'$ such that $X_t = X_{t'} \setminus \{w\}$ for some vertex $w \in X_{t'}$
		\item \textbf{Join node}: a node $t$ with exactly two children $t_1$, $t_2$ such that $X_t = X_{t_1} = X_{t_2}$.
	\end{itemize}
\end{itemize}

\chapter{Spanning Star Forest Problem}\label{r:losers}

For a given graph $G$, we say that $S$ is a \ssf{} if every connected component $C$ is a star. In \ssfp{} given a graph $G$, the objective is to determine whether there exists a \ssf{}.

It turns out that the problem formulated in such a way is relatively simple. Although, various parametrizations described in this paper make it more complex. The following lemma easily clarifies all the concerns about it's hardness.

\begin{lemma}\label{SSF lemma}
 A graph $G$ has a \ssf{} if and only if it does not contain any isolated vertices.
\end{lemma}

\begin{proof}
	If $G$ has a \ssf{} $S$, then trivially for all $v \in V(G)\ 1 \leq deg_S(v) \leq deg_G(v)$. Thus, none of the vertices is isolated.
	
	For the opposite direction, we prove the lemma by induction on $|V(G)|$. Assume $|V(G)|=2$. The statement trivially holds because a graph representing an edge is a correct \ssf. Let $|V(G)| >2$. Suppose that there does not exist a vertex $v$ such that $G \setminus \{v\}$ has no isolated vertices. Then, it holds that for all $v \in V(G)\ deg_G(v)=1$ so $G$ itself	is a correct \ssf{}. Now, suppose that $v$ is a vertex such that $G \setminus \{v\}$ has no isolated vertices. From the inductive assumption, let $S$ be a \ssf{} of a graph $G \setminus \{v\}$, $u$ be a vertex such that $(u,v) \in E(G)$ and $w$ be a vertex such that $w \in N_S(u)$. Consider the 3 following cases:
	\begin{enumerate}
		\item $deg_S(u) > 1$. Then, $S'=\big(V(S) \cup \{v\}, E(S) \cup \{(u,v)\}\big)$ is a correct solution for graph $G$.
		\item $deg_S(u) = deg_S(w) = 1$. Then, $S'=\big(V(S) \cup \{v\}, E(S) \cup
		(u,v)\big)$ is a correct solution for graph $G$.
		\item $deg_S(w) > 1$. Then, $S'=\big(V(S) \cup \{v\}, 
		(E(S) \cup \{(u,v)\}) \setminus \{(u,w)\}\big)$ is a correct solution for graph $G$.
	\end{enumerate}
		Observe that in graph $G$ there are no isolated vertices. Thus, one can always extend a solution inductively.
	
\end{proof}
	Application of Lemma \ref{SSF lemma} yields the following result for \ssfp{}.

\begin{theorem}
	Decision version of \ssfp{} can be solved in linear time.
\end{theorem}

\begin{proof}
	Given an input $G = (V,E)$ the answer is YES if for all $v \in V(G)\ deg_G(v) \neq 0$ and NO otherwise.
\end{proof}

\section{Obtaining a solution}

In this section we focus on obtaining an arbitrary solution for a given instance of \ssfp{}. Firstly, let us introduce 2 claims in order to normalize the instance and make the algorithm look more clear.

\begin{claim} \label{SSF sum}
	Family of disjoint {\normalfont Spanning Star Forests} is a \ssf{}.
\end{claim}

\begin{claim} \label{Spanning tree SSF}
	$G$ has a \ssf{} if and only if it's spanning tree $T$ has.
\end{claim}

The first claim can be trivially proven by the definition of \ssfp{} while the second one follows directly from Lemma \ref{SSF lemma}. Equipped with this information, all that is left to do, is to design an algorithm which solves \ssfp{} for trees.

\begin{algorithm}\label{alg1}
	\KwData{Graph G}
	\KwResult{Spanning Star Forest of T}
	$T \leftarrow$ $SpanningTree(G)$\;
	$S \leftarrow$ $\emptyset$\;
	\For{v: postorder(T) and v $\notin$ V(S)}{
		\eIf{v is not a root}{
			S $\leftarrow S \cup \{(u,v)\}$ where $u = parent(v)$
		}{
			S $\leftarrow S \cup \{(u,v)\}$ where $u$ is any of the root's children
		}
	}
	\Return S
	\caption{Obtaining a Spanning Star Forest from a tree.}
\end{algorithm}

\begin{lemma}\label{alg1 correctness}
	Algorithm \ref{alg1} is correct.
\end{lemma}

\begin{proof}
	Assume contrary, that the algorithm yields an incorrect solution $S$. Consider
	the first case: a path $(u,v),(v,w),(w,z)$ exists in $S$ where $u$ is $v$'s 
	child, $v$ is $w$'s child and $w$ is $z$'s child. But, if $u$ is $w$'s
	grandchild and $(u,v),(v,w) \in S$, then it means that $w$ is a root. 
	Contradiction because $w$ cannot be $z$'s child.
	Now, suppose the alternative relationship: $u$ is $v$'s child, $v$ and $z$ are
	$w$'s children. Provided that vertices were visited in postorder, edge $(v,w)$
	should not have been added because $v$ was introduced by $u$ and $w$ was
	introduced by $z$.
\end{proof}

\begin{theorem}
	A solution for Spanning Star Forest Problem can be found in linear time.
\end{theorem}

\begin{proof}
	Spanning tree of any graph can be found in linear time. The main loop has $n$ 
	iterations (every vertex is visited once), each of which takes constant time.
	Thus, the total runtime is linear.
\end{proof}

\section{Spanning Star Forest parameterized by the number of stars}

In \kssf{}, given a graph $G$ and a natural number $k$, the objective is to determine whether there exists a \ssf{} $S$ such that the number of connected components is at most $k$.

It is natural to ask whether one can find a solution that minimizes the number of connected components. The problem formulated in that way looks slightly different than the previous one. From the other hand, the problem resembles \domsetp{}, which is defined as follows:

\begin{defi}
	\domsetp: Given a graph $G$ and a positive integer $k$ find a set $D$ such that $|D| \leq k$ and every vertex from the graph is adjacent to one of the vertices from $D$.
\end{defi}

It turns out, that the second comparison is true and \kssf{} is NP-Complete. But, before we begin, let us introduce one more definition and a lemma that supports a reduction.

\begin{defi}
	Given an instance $(G,k)$ of Dominating Set Problem that does not contain any isolated vertices and a solution $D$, a dominating mapping is a function $m:V(G) \setminus D \rightarrow D$ such that satisfies $(x,m(x)) \in E(G)$ for all $x \in Dom(m)$.
\end{defi}

\begin{lemma}\label{dom mapping}
	Let $(G,k)$ be an instance of \domsetp{} without isolated vertices and let $D$ be a solution of minimal size. Then, there exists a dominating mapping $m$ such that $m$ is surjective.
\end{lemma}

\begin{proof}
	Suppose contrary that such a mapping does not exist i.e. for every mapping $m$ there exists a vertex $v \in D$ such that $v \notin im(m)$. Let us break the proof into 4 cases:
	\begin{enumerate}
		\item Suppose $N_G(v) = \emptyset$. Contradiction, $G$ has no isolated vertices.
		\item Suppose $u \in N_G(v) \cap D$. Contradiction, $D$ was said to be a solution of minimal size whereas $D \setminus \{u\}$ is a valid, smaller solution.
		\item Suppose $u \in N_{G}(v) \setminus D$ and $w \in N_G(u) \cap im(m)$. If $|m^{-1}(w)|=1$, then $((D \setminus \{v,w\}) \cup u)$ is a valid, smaller solution for a graph $G$. Contradiction.
		\item Suppose $u \in N_{G}(v) \setminus D$ and $w \in N_G(u) \cap im(m)$. If $|m^{-1}(w)| > 1$ then a mapping:
		\begin{equation*}
			m'(x) = \begin{cases}
			v, & \text{if x = u} \\
			m(x), &\text{otherwise} \\
			\end{cases}
		\end{equation*}
		is a valid mapping that satisfies $im(m) \subset im(m')$. Thus, one can create a new mapping $m''$ inductively such that $m''$ is surjective. Contradiction, we assumed that no such mapping exists. 
	\end{enumerate}
	
	Since all the possible cases led to a contradiction, we may conclude that there exists a dominating mapping $f$ such that $f$ is surjective.  
\end{proof}

Armed with the lemma, we are ready to prove the main theorem of the section.

\begin{theorem}\label{dom ssf}
	\kssf{} is NP-Complete.
\end{theorem}

\begin{proof}
	Membership in NP: given an oracle $(O,k)$, we check whether the number of	components in $O$ is less than $k$ and whether every connected component	forms a star. The task can be easily done in polynomial time.\\
	
	We show hardness by a reduction from \domsetp{} that completes the proof. Let $(G,k)$ be an instance of it. We create a graph	$G'$ as follows: for every isolated vertex $v \in V(G)$ introduce a vertex $v'$ and an edge $(v,v')$. Now, we claim that $(G,k)$ is a YES-instance for \domsetp{} if and only if $(G'.k)$ is a YES-instance for \kssf{}. 
	
	The backward implication is simple. Suppose $S$ is a solution for $(G',k)$. We claim that a set $D$ representing centers of stars is a correct \domset{}. Obviously $|D| \leq k$ because there are at most $k$ connected components. Every vertex from $G'$ is adjacent to one of the centers. If there exists a vertex $v' \in D$ such that $v' \notin V(G)$ we transform the solution as follows: $D := (D \setminus \{v'\})\cup \{v\}$. 
	
	To prove the forward implication, let $D$ be a solution of minimal size for $(G,k)$. Obviously, $D$ is also a minimal solution for a graph $G'$. Thus, by lemma \ref{dom mapping}. there exists a mapping $m$ that is surjective. Now, we claim that a graph $S=(V(G'),\{(x,m(x)): x \in Dom(m)\})$ is a correct solution for \ssfp{}. Trivially, there are no isolated vertices in $S$. Moreover, there is no path of length 4 because $S$ consists of edges $(v,u)$ such that $v \in D$, $u \notin D$ and for all $u \in V(S) \setminus D$ $deg_S(u)=1$.
	
\end{proof}

The theorem implies that \kssf{} is as hard as \domsetp{}. Thus, we can immediately obtain the following corollary.

\begin{corollary}
	\kssf{} is {\normalfont W[2]-complete}.
\end{corollary}

The problems look so similar that one could ask whether the reverse reduction is true. Indeed, with a small twist to the previous idea one can prove the reverse reduction instantly.  

\begin{theorem}\label{ssf dom}
	There exists a reduction from \kssf{} to \domsetp{}.
\end{theorem}

\begin{proof}
	Let $(G,k)$ be an instance of \ssfp{}. We create an instance $(G',k')$ for \domset{} as follows: let $G'=G$ and if $G$ contains an isolated vertex, then $k'=0$. Otherwise, the value remains the same. Now, we claim that $(G,k)$ is a YES-instance for \ssfp{} if and only if $(G',k')$ is a YES-instance for \domset.
	
	To prove the following reduction one can use the method which was described in Theorem \ref{ssf dom} with a little remark: if an instance $(G,k)$ contains an isolated vertex, then obviously it is a NO-instance for \ssfp{} and so is $(G',k')$ for \domset{} because $G'$ is not an empty graph.
\end{proof}

One can observe now the immediate corollary of the theorem \ref{dom ssf} and theorem \ref{ssf dom}.

\begin{corollary}
	Every theorem is true for \domsetp{} if and only if it is true for a \kssf{}.
\end{corollary}

As an example, the following theorem described in 
%TODO dodaj ref%
can be transfered to \kssf{}.

\begin{theorem}
	Unless \cnfsat{} can be solved in time $\mathcal{O}^*((2-\epsilon')^n)$ for some $\epsilon' > 0$ there do not exist constant $\epsilon > 0$, $k \geq 3$ and an algorithm solving \domsetp{} parameterized by the number of stars in time $\mathcal{O}^*(N^{k-\epsilon})$, where $N$ is the number of vertices of the input graph. 
\end{theorem}

\chapter{Spanning Star Forest Extension Problem}

In this chapter, we significantly change the problem. Let $G$ be a graph and $F \subseteq E(G)$. In the \ssfe{} the question that we want to answer now is that whether there exists a \ssf{} $S$ such that $F \subseteq E(S)$. Hardness of the problem lays in deciding which end of every isolated edge is a center and which one is a ray. We used three different parameters: number of isolated edges, number of non-isolated edges and treewidth. 

\section{General overview}

\subsection{Instance normalization} 

Observe that a star is a primitive structure. The star's maximal radius is equal to 3. It means that we can look at the problem rather locally than globally. Notice that this time we do not have any limit on the number of connected components. As it was said before, the hardness of the problem lays in choosing which of the endings of an isolated edge is a center. Thus, it might be possible to normalize instances i.e. try to remove vertices that are "far enough" from isolated vertices.

Let $(G, F)$ be an arbitrary instance of \ssfe{}. Firstly, consider trivial cases.

\begin{enumerate}[leftmargin=*,label=\textbf{Reduction \arabic{enumi}}]
	\item If graph $G$ contains an isolated vertex, it is a NO-instance.
	\item If in graph $G$ there exists a path of size at least 3 made from isolated edges, it is a NO-instance.
\end{enumerate}

Suppose that isolated edges form a star of size at least 3. Then, the center is already set. Thus, we can remove from the instance all the vertices that are adjacent to the pre-created centers.
\begin{enumerate}[leftmargin=*,label=\textbf{Reduction \arabic{enumi}},resume]
	\item $C = \{v: |N_G(v) \cap V(F)| > 1\}$. Update $G= G \setminus (N_G(C) \cup C)$.
\end{enumerate}

Now, let $V_P = \{v: (v,u) \in (E(G) \setminus F) \text{ and } v \notin V(F)\}$ and $V_{NP} = V(G) \setminus V_P$. Finally, $G_{NP} = G[V_{NP}]$ and $G_P = G[V_P]$. Notice an immediate consequence of the partitioning of graph $G$.

\begin{claim}\label{gp cut}
	$G_P$ always has a solution.
\end{claim}

To prove the claim we can apply lemma \ref{SSF lemma}. $G_P$ does not have any isolated edges nor isolated vertices. All that is left to do is to prove that edges between $G_P$ and $G_{NP}$, that were lost during partitioning, does not have any effect on the solution. The following theorem proves the intuition.

\begin{lemma}
	An instance $(G,F)$ has a solution if and only if $(G_{NP},F)$ has one.
\end{lemma}

\begin{proof}
	The backward implication is trivial. Suppose $S$ is a solution for an instance $(G_{NP},F)$. We can partition $G$ into $G_P$ and $G_{NP}$ and find a solution, say $S'$, for a graph $G_P$. Then, $S \cup S'$ is a correct solution for $G$.
	
	Now, consider the forward implication. Let $S$ be a solution for an instance $(G, F)$. Assume contrary that there exists a vertex $v \in V(G_{NP})$ does not belong to any star. Trivially, vertices from $V(F)$ are covered. Thus, $v \in V(G_{NP}) \setminus V(F)$. If $v \in V(G_{NP}) \setminus V(F)$ it follows that for all $(v,u) \in E(G)$ $(v,u) \in E(G_{NP}))$. If it was not true, the vertex $v$ would have been place in the graph $G_P$. Thus, there exists an edge $(v,u) \in E(S)$ such that $(v,u) \in E(G_{NP})$. Contradiction.
	
\end{proof}
\noindent
Ultimately, we can state the last rule.
\begin{enumerate}[leftmargin=*,label=\textbf{Reduction \arabic{enumi}},resume]
	\item Update $G = G \setminus G_P$.
\end{enumerate}

\subsection{NP-completeness}

After exhaustive application of rules, any graph has a simple structure. Let $(G,F)$ be an instance of \ssfe{} after normalization. Then, $G$ consists of vertices of two types: ones that have only edges to vertices from $V(F)$ and ones that have exactly one isolated edge (and potentially many non-isolated). Such a representation substantially simplifies further investigations. Indeed, we can prove that the problem parameterized by the number of isolated edges is NP-complete.



\begin{theorem}
	\ssfe{} is NP-complete.
\end{theorem}

\begin{proof}
	Membership in NP: Given a solution $S$ for instance $(G,F)$ check whether $F \subseteq S$ and whether $S$ is a correct \ssf{}.
	\\\\
	To prove hardness, we show a reduction from \tsat{}. Let $\phi$ be an arbitrary instance. We create a graph $G$ as follows: for every variable $x_i$ we introduce vertices $x_i,\neg x_i$ and an isolated edge $(x_i,\neg x_i)$. For every clause $c_i$ we introduce a vertex $c_i$. For every $x_k$ (symmetrically $\neg x_k$) we introduce an edge $(x_k,c_p)$ if and only if $x_k$ is present in p'th clause. Now, we claim that $\phi$ is satisfiable if and only if $(G, F)$ has a \ssf{}. \\
	Backward implication: Let $S$ be a solution for $(G,F)$. Then, a set of centers is a correct evaluation that satisfies the formula $\phi$ because every clause $S$ has a witness.\\
	To prove the forward implication, assume there exists an evaluation $\sigma$ of variables that satisfies the formula. Let $C = \{l_i: \sigma(l_i) = 1\} $. Note that $C$ contains either $x_i$ or $\neg x_i$. Now, let us construct a solution $S$. Firstly, include all the isolated edges. Then, for every vertex representing a clause $c_i$ take a random $l_j$ such that $l_j \in N(c_i) \cap C$ and include edge $(c_i, l_j)$ into the solution. The operation is safe. All sets $ N(c_i) \cap C$ are nonempty because there exists a witness $l_k$ that satisfies the clause and there exists an edge between a literal and a clause.
\end{proof}

Surprisingly, \tsat{} is trivially encoded in \ssfe{}.

\lemma{Any CNF formula can be represented as a formula where each clause has size at most 3} \label{cnf to 3sat}
\begin{lemma}
	There exists a reduction from \ssfe{} to \tsat.
\end{lemma}

\begin{proof}
	Let $(G,F)$ be an arbitrary normalized instance of \ssfe{}. We create a formula $\phi$ as follows: for every isolated edge $(u,v)$ we introduce literals $x$ and $\neg x$. For every vertex $v \in V(G)\setminus V(F)$ we introduce a clause $c_v$. Literal $l$ is present in a clause $c_j$ if and only if there exists an edge between vertices corresponding to $c_i$ and $l$. Application of lemma \ref{cnf to 3sat} to $\phi$ produces a formula $\phi'$ that is a correct instance of \tsat.
	
	Proof of correctness was previously described in the theorem. We omit it for the sake of clarity.
\end{proof}

\section{Parametrization by the number of isolated edges}

\section{Parametrization by the number of non-isolated edges}

\section{Parametrization by treewidth}

\begin{thebibliography}{99}
\addcontentsline{toc}{chapter}{Bibliografia}

\bibitem[Bea65]{beaman} Juliusz Beaman, \textit{Morbidity of the Jolly
    function}, Mathematica Absurdica, 117 (1965) 338--9.

\bibitem[Blar16]{eb1} Elizjusz Blarbarucki, \textit{O pewnych
    aspektach pewnych aspektów}, Astrolog Polski, Zeszyt 16, Warszawa
  1916.

\bibitem[Fif00]{ffgg} Filigran Fifak, Gizbert Gryzogrzechotalski,
  \textit{O blabalii fetorycznej}, Materiały Konferencji Euroblabal
  2000.

\bibitem[Fif01]{ff-sr} Filigran Fifak, \textit{O fetorach
    $\sigma$-$\rho$}, Acta Fetorica, 2001.

\bibitem[Głomb04]{grglo} Gryzybór Głombaski, \textit{Parazytonikacja
    blabiczna fetorów --- nowa teoria wszystkiego}, Warszawa 1904.

\bibitem[Hopp96]{hopp} Claude Hopper, \textit{On some $\Pi$-hedral
    surfaces in quasi-quasi space}, Omnius University Press, 1996.

\bibitem[Leuk00]{leuk} Lechoslav Leukocyt, \textit{Oval mappings ab ovo},
  Materiały Białostockiej Konferencji Hodowców Drobiu, 2000.

\bibitem[Rozk93]{JR} Josip A.~Rozkosza, \textit{O pewnych własnościach
    pewnych funkcji}, Północnopomorski Dziennik Matematyczny 63491
  (1993).

\bibitem[Spy59]{spyrpt} Mrowclaw Spyrpt, \textit{A matrix is a matrix
    is a matrix}, Mat. Zburp., 91 (1959) 28--35.

\bibitem[Sri64]{srinis} Rajagopalachari Sriniswamiramanathan,
  \textit{Some expansions on the Flausgloten Theorem on locally
    congested lutches}, J. Math.  Soc., North Bombay, 13 (1964) 72--6.

\bibitem[Whi25]{russell} Alfred N. Whitehead, Bertrand Russell,
  \textit{Principia Mathematica}, Cambridge University Press, 1925.

\bibitem[Zen69]{heu} Zenon Zenon, \textit{Użyteczne heurystyki
    w~blabalizie}, Młody Technik, nr~11, 1969.

\end{thebibliography}

\end{document}


%%% Local Variables:
%%% mode: latex
%%% TeX-master: t
%%% coding: latin-2
%%% End:
