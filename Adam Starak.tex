%
% Niniejszy plik stanowi przykład formatowania pracy magisterskiej na
% Wydziale MIM UW.  Szkielet użytych poleceń można wykorzystywać do
% woli, np. formatujac wlasna prace.
%
% Zawartosc merytoryczna stanowi oryginalnosiagniecie
% naukowosciowe Marcina Wolinskiego.  Wszelkie prawa zastrzeżone.
%
% Copyright (c) 2001 by Marcin Woliński <M.Wolinski@gust.org.pl>
% Poprawki spowodowane zmianami przepisów - Marcin Szczuka, 1.10.2004
% Poprawki spowodowane zmianami przepisow i ujednolicenie 
% - Seweryn Karłowicz, 05.05.2006
% Dodanie wielu autorów i tłumaczenia na angielski - Kuba Pochrybniak, 29.11.2016

% dodaj opcję [licencjacka] dla pracy licencjackiej
% dodaj opcję [en] dla wersji angielskiej (mogą być obie: [licencjacka,en])
\documentclass[en]{pracamgr}

% Dane magistranta:
\autor{Adam Starak}{361021}

% TODO[Dodaj tytuł]
\title{Title in English}
\titlepl{Tytuł po polsku}

%\tytulang{An implementation of a difference blabalizer based on the theory of $\sigma$ -- $\rho$ phetors}

%kierunek: 
% - matematyka, informacyka, ...
% - Mathematics, Computer Science, ...
\kierunek{Computer Science}

% informatyka - nie okreslamy zakresu (opcja zakomentowana)
% matematyka - zakres moze pozostac nieokreslony,
% a jesli ma byc okreslony dla pracy mgr,
% to przyjmuje jedna z wartosci:
% {metod matematycznych w finansach}
% {metod matematycznych w ubezpieczeniach}
% {matematyki stosowanej}
% {nauczania matematyki}
% Dla pracy licencjackiej mamy natomiast
% mozliwosc wpisania takiej wartosci zakresu:
% {Jednoczesnych Studiow Ekonomiczno--Matematycznych}

% \zakres{Tu wpisac, jesli trzeba, jedna z opcji podanych wyzej}

% Praca wykonana pod kierunkiem:
% (podać tytuł/stopień imię i nazwisko opiekuna
% Instytut
% ew. Wydział ew. Uczelnia (jeżeli nie MIM UW))
\opiekun{dr Michał Pilipczuk\\
  Instytut Informatyki\\
  }

% miesiąc i~rok:
\date{May 2017}

%Podać dziedzinę wg klasyfikacji Socrates-Erasmus:
\dziedzina{ 
%11.0 Matematyka, Informatyka:\\ 
%11.1 Matematyka\\ 
%11.2 Statystyka\\ 
11.3 Informatyka\\ 
%11.4 Sztuczna inteligencja\\ 
%11.5 Nauki aktuarialne\\
%11.9 Inne nauki matematyczne i informatyczne
}

%Klasyfikacja tematyczna wedlug AMS (matematyka) lub ACM (informatyka)
%TODO - dodać klasyfikację
\klasyfikacja{D. Software\\
  D.127. Blabalgorithms\\
  D.127.6. Numerical blabalysis}

%TODO - dodać słowa kluczowe]
% Słowa kluczowe:
\keywords{parameterized algorithm}

% Tu jest dobre miejsce na Twoje własne makra i~środowiska:

\usepackage{chngcntr}
\usepackage{amsthm}
\usepackage[]{algorithm2e}

\newtheorem{defi}{Definition}
\newtheorem{theorem}{Theorem}
\newtheorem{lemma}{Lemma}
\newtheorem{claim}{Claim}

\counterwithin{theorem}{chapter}
\counterwithin{defi}{chapter}
\counterwithin{lemma}{chapter}
\counterwithin{claim}{chapter}

% koniec definicji

\begin{document}
\maketitle

%tu idzie streszczenie na strone poczatkowa
%TODO - dodaj abstract
\begin{abstract}
  W~pracy przedstawiono prototypową implementację blabalizatora
  różnicowego bazującą na teorii fetorów $\sigma$-$\rho$ profesora
  Fifaka.  Wykorzystanie teorii Fifaka daje wreszcie możliwość
  efektywnego wykonania blabalizy numerycznej.  Fakt ten stanowi
  przełom technologiczny, którego konsekwencje trudno z~góry
  przewidzieć.
\end{abstract}

\tableofcontents
%\listoffigures
%\listoftables

\chapter*{Introduction}
\addcontentsline{toc}{chapter}{Introduction}

Blabalizator różnicowy jest podstawowym narzędziem blabalii
fetorycznej.  Dlatego naukowcy z~całego świata prześcigają się
w~próbach efektywnej implementacji.  Opracowana przez prof. Fifaka
teoria fetorów $\sigma$-$\rho$ otwiera w~tej dziedzinie nowe
możliwości.  Wykorzystujemy je w~niniejszej pracy.

\chapter{Basic definitions}\label{r:pojecia}

\section{Structures}

A simple graph $G$ is a pair $(V,E)$ where $V$ denotes a set of vertices
and $E$ denotes a set of undirected edges. Let $deg_G(v)$ denote a degree
of vertex v in graph $G$. Let $G \setminus \{v\}$ be the abbreviation for
$G'=(V(G) \setminus \{v\}, E(G) \setminus \{(u,v): u \in V(G)\})$.A \emph{tree} $T$ is a graph where two
vertices are connected by excatly one path. A \emph{spanning tree} $T$ of a
graph $G$ is a graph which includes all of the vertices of $G$, with minimum
possible number of edges. A \emph{star} $S$ is a tree of size at least 2
for which at most 1 vertex has a degree greater than $1$.


\section{Parameterized complexity}

\begin{defi}\label{Parameterized problem}
	Parameterized problem
\end{defi}

\begin{defi}\label{FPT algorithm}
	FPT algorithm
\end{defi}

\begin{defi}\label{Kernel}
	Kernel
\end{defi}

\begin{defi}\label{Kernelization algorithm}
	Kernelization algorithm
\end{defi}

\section{Graph decomposition}

\begin{defi}\label{Pathwidth}
	Path decomposition and pathwidth
\end{defi}

\begin{defi}\label{Treewidth}
	Tree decomposition and treewidth
\end{defi}

\begin{defi}\label{nice tree decomposition}
	Nice tree decomposition
\end{defi}

\chapter{Spanning Star Forest Problem}\label{r:losers}

For a given graph $G$, we say that $S$ is a \emph{Spanning Star Forest}
if every connected component $C$ is a star. In the 
\emph{Spanning Star Forest Problem} given a graph $G$, the objective is
to determine whether there exists a \emph{Spanning Star Forest}. 

It turns out that the problem formulated in such a way is relatively 
simple. Although, various parametrizations described in this paper 
make it more complex. The following lemma easily clarifies all the 
concerns about hardness.

\begin{lemma}\label{SSF lemma}
 A graph $G$ has a Spanning Star Forest if and only if it does not contain
 any isolated vertices.
\end{lemma}

\begin{proof}
	If $G$ has a Spanning Star Forest $S$, then trivially 
	$\forall_{v \in V(G)}\ 1 \leq deg_S(v) \leq deg_G(v)$. Thus, none of the
	vertices is isolated.
	
	%TODO |V(G)|=2 wychodzi poza linię.
	For the opposite direction, we prove the lemma by induction on $|V(G)|$.
	Assume ${|V(G)|=2}$. The statement trivially holds because a graph representing
	an edge is a correct Spanning Star Forest. Let $|V(G)| >2$ and let $v$ be 
	a vertex such that $G \setminus \{v\}$ has no isolated vertices. (If no such
	vertex exists, it holds that $\forall_{v \in V(G)}\ deg_G(v)=1$ so $G$ itself
	is a correct Spanning Star Forest) From the inductive assumption, let $S$ be a Spanning Star
	Forest of a graph $G \setminus \{v\}$, $u$ be a vertex such that $(u,v)
	\in E(G)$ and $w$ be a vertex such that $w \in N_S(u)$. Consider the 3
	following cases:
	\begin{enumerate}
		\item $deg_S(u) > 1$. Then, $S'=\big(V(S) \cup \{v\}, E(S) \cup \{(u,v)\}\big)$ is a correct solution for graph $G$.
		\item $deg_S(u) = deg_S(w) = 1$. Then, $S'=\big(V(S) \cup {v}, E(S) \cup
		(u,v)\big)$ is a correct solution for graph $G$.
		\item $deg_S(w) > 1$. Then, $S'=\big(V(S) \cup \{v\}, 
		(E(S) \cup \{(u,v)\}) \setminus \{(u,w)\}\big)$ is a correct solution for graph $G$.
	\end{enumerate}
		Observe that in graph $G$ there are no isolated vertices. Thus, one can
		always extend a solution inductively.
	
\end{proof}
	Application of Lemma \ref{SSF lemma} yields the following result for Spanning
	Star Forest Problem:

\begin{theorem}
	Decision version of Spanning Star Forest Problem can be solved in linear
	time.
\end{theorem}

\begin{proof}
	Given an input $G = (V,E)$ the answer is YES if $\forall_{v \in V(G)}\ 
	deg_G(v) \neq 0$ and NO otherwise.
\end{proof}

\section{Obtaining a solution}

In this section the focus will be set on obtaining an arbitrary solution for
a given instance of \emph{Spanning Star Forest Problem}. Firstly, let's
introduce 2 claims in order to normalize the instance and make the algorithm
look more clear.

\begin{claim} \label{SSF sum}
	Family of disjoint Spanning Star Forests is a Spanning Star Forest.
\end{claim}

\begin{claim} \label{Spanning tree SSF}
	$G$ has a Spanning Star Forest if and only if it's spanning tree $T$ has.
\end{claim}

The first claim can be trivially proven by the definition of 
\emph{Spanning Star Forest Problem} while the second one follows directly from
Lemma \ref{SSF lemma}. Equipped with this information, all that is left to do,
is to design an algorithm which solves \emph{Spanning Star Forest Problem} for 
trees.

\begin{algorithm}\label{alg1}
	\KwData{Graph G}
	\KwResult{Spanning Star Forest of T}
	$T \leftarrow$ $SpanningTree(G)$\;
	$S \leftarrow$ $\emptyset$\;
	\For{v: postorder(T) and v $\notin$ V(S)}{
		\eIf{v is not a root}{
			S $\leftarrow S \cup \{(u,v)\}$ where $u = parent(v)$
		}{
			S $\leftarrow S \cup \{(u,v)\}$ where $u$ is any of the root's children
		}
	}
	\Return S
	\caption{Obtaining a Spanning Star Forest from a tree.}
\end{algorithm}

\begin{lemma}\label{alg1 correctness}
	Algorithm \ref{alg1} is correct.
\end{lemma}

\begin{proof}
	Assume contrary, that the algorithm yields an incorrect solution $S$. Consider
	the first case: a path $(u,v),(v,w),(w,z)$ exists in $S$ where $u$ is $v$'s 
	child, $v$ is $w$'s child and $w$ is $z$'s child. But, if $u$ is $w$'s
	grandchild and $(u,v),(v,w) \in S$ it means that $w$ is a root. Contradiction
	because	$w$ cannot be $z$'s child.
	Now, suppose the alternative relationship: $u$ is $v$'s child, $v$ and $z$ are
	$w$'s children. Provided that vertices were visited in postorder, edge $(v,w)$
	should not have been added because $v$ was introduced by $u$ and $w$ was
	introduced by $z$.
\end{proof}

\begin{theorem}
	A solution for Spanning Star Forest Problem can be found in linear time.
\end{theorem}

\begin{proof}
	Spanning tree of any graph can be found in linear time. The loop has $n$ 
	iterations (every vertex is visited once), each of which takes constant time.
	Thus, the total runtime is linear.
\end{proof}

\section{Spanning Star Forest parameterized by the number of stars}

In \emph{Spanning Star Forest Problem} parameterized by the number of
stars, given a graph $G$ and a natural number $k$, the objective is to
determine whether there exists a \emph{Spanning Star Forest} $S$ such that
the number of connected components is less than $k$.

It is natural to ask whether one can find a solution that minimizes the number
of connected components. Even though the problem looks slightly different
than the previous one, \emph{Spanning Star Forest} parameterized by the 
number of stars is NP-Complete. The following theorem proves the statement:

\begin{theorem}
	Spanning Star Forest Parameterized by the number of stars is NP-Complete.
\end{theorem}



\begin{proof}
	Membership in NP: given an oracle $(O,k)$, check whether the number of
	components in $O$ is less than $k$ and whether every connected component 
	forms a star. The task can be easily done in polynomial time.
	
	A reduction from \emph{Dominating Set} completes the proof. Here, an input
	is a graph $G$ and an integer $k$ and the task is to find a set 
	$S \subseteq V(G)$ such that $|S|\leq k$ and:
	
	$$\bigcup_{v \in S} \big(N_G(v) \cup \{v\}\big) = V(G)$$.
	
	Now, given an instance $(G,k)$ of \emph{Dominating Set} let $I = \{v: v$ is 
	isolated in $G\}$. All that remains, is to prove that $(G \setminus I,
	k-|I|)$ is a YES-instance for \emph{Spanning Star Forest Problem} 
	parameterized by the number of stars if and only if $(G,k)$ is a YES-instance 
	for \emph{Dominating Set Problem}. The forward implication is simple.
	
	
\end{proof}


\begin{lemma}
	There exists a reduction from Spanning Star Forest parameterized by the
	number of stars to Dominating Set.
\end{lemma}


\begin{thebibliography}{99}
\addcontentsline{toc}{chapter}{Bibliografia}

\bibitem[Bea65]{beaman} Juliusz Beaman, \textit{Morbidity of the Jolly
    function}, Mathematica Absurdica, 117 (1965) 338--9.

\bibitem[Blar16]{eb1} Elizjusz Blarbarucki, \textit{O pewnych
    aspektach pewnych aspektów}, Astrolog Polski, Zeszyt 16, Warszawa
  1916.

\bibitem[Fif00]{ffgg} Filigran Fifak, Gizbert Gryzogrzechotalski,
  \textit{O blabalii fetorycznej}, Materiały Konferencji Euroblabal
  2000.

\bibitem[Fif01]{ff-sr} Filigran Fifak, \textit{O fetorach
    $\sigma$-$\rho$}, Acta Fetorica, 2001.

\bibitem[Głomb04]{grglo} Gryzybór Głombaski, \textit{Parazytonikacja
    blabiczna fetorów --- nowa teoria wszystkiego}, Warszawa 1904.

\bibitem[Hopp96]{hopp} Claude Hopper, \textit{On some $\Pi$-hedral
    surfaces in quasi-quasi space}, Omnius University Press, 1996.

\bibitem[Leuk00]{leuk} Lechoslav Leukocyt, \textit{Oval mappings ab ovo},
  Materiały Białostockiej Konferencji Hodowców Drobiu, 2000.

\bibitem[Rozk93]{JR} Josip A.~Rozkosza, \textit{O pewnych własnościach
    pewnych funkcji}, Północnopomorski Dziennik Matematyczny 63491
  (1993).

\bibitem[Spy59]{spyrpt} Mrowclaw Spyrpt, \textit{A matrix is a matrix
    is a matrix}, Mat. Zburp., 91 (1959) 28--35.

\bibitem[Sri64]{srinis} Rajagopalachari Sriniswamiramanathan,
  \textit{Some expansions on the Flausgloten Theorem on locally
    congested lutches}, J. Math.  Soc., North Bombay, 13 (1964) 72--6.

\bibitem[Whi25]{russell} Alfred N. Whitehead, Bertrand Russell,
  \textit{Principia Mathematica}, Cambridge University Press, 1925.

\bibitem[Zen69]{heu} Zenon Zenon, \textit{Użyteczne heurystyki
    w~blabalizie}, Młody Technik, nr~11, 1969.

\end{thebibliography}

\end{document}


%%% Local Variables:
%%% mode: latex
%%% TeX-master: t
%%% coding: latin-2
%%% End:
