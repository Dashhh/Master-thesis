 %
% Niniejszy plik stanowi przykład formatowania pracy magisterskiej na
% Wydziale MIM UW.  Szkielet użytych poleceń można wykorzystywać do
% woli, np. formatujac wlasna prace.
%
% Zawartosc merytoryczna stanowi oryginalnosiagniecie
% naukowosciowe Marcina Wolinskiego.  Wszelkie prawa zastrzeżone.
%
% Copyright (c) 2001 by Marcin Woliński <M.Wolinski@gust.org.pl>
% Poprawki spowodowane zmianami przepisów - Marcin Szczuka, 1.10.2004
% Poprawki spowodowane zmianami przepisow i ujednolicenie 
% - Seweryn Karłowicz, 05.05.2006
% Dodanie wielu autorów i tłumaczenia na angielski - Kuba Pochrybniak, 29.11.2016

% dodaj opcję [licencjacka] dla pracy licencjackiej
% dodaj opcję [en] dla wersji angielskiej (mogą być obie: [licencjacka,en])
\documentclass[en]{pracamgr}

% Dane magistranta:
\autor{Adam Starak}{361021}

\title{Application of parameterized techniques to finding spanning star forests in graphs}
\titlepl{Zastosowanie technik algorytmów parametryzowanych dla problemu znajdowania gwiazd rozpinających grafy}

%\tytulang{An implementation of a difference blabalizer based on the theory of $\sigma$ -- $\rho$ phetors}

%kierunek: 
% - matematyka, informacyka, ...
% - Mathematics, Computer Science, ...
\kierunek{Computer Science}

% informatyka - nie okreslamy zakresu (opcja zakomentowana)
% matematyka - zakres moze pozostac nieokreslony,
% a jesli ma byc okreslony dla pracy mgr,
% to przyjmuje jedna z wartosci:
% {metod matematycznych w finansach}
% {metod matematycznych w ubezpieczeniach}
% {matematyki stosowanej}
% {nauczania matematyki}
% Dla pracy licencjackiej mamy natomiast
% mozliwosc wpisania takiej wartosci zakresu:
% {Jednoczesnych Studiow Ekonomiczno--Matematycznych}

% \zakres{Tu wpisac, jesli trzeba, jedna z opcji podanych wyzej}

% Praca wykonana pod kierunkiem:
% (podać tytuł/stopień imię i nazwisko opiekuna
% Instytut
% ew. Wydział ew. Uczelnia (jeżeli nie MIM UW))
\opiekun{dr Michał Pilipczuk\\
  Institute of Informatics\\
  }

% miesiąc i~rok:
\date{\monthyeardate\today}

%Podać dziedzinę wg klasyfikacji Socrates-Erasmus:
\dziedzina{ 
%11.0 Matematyka, Informatyka:\\ 
%11.1 Matematyka\\ 
%11.2 Statystyka\\ 
11.3 Informatyka\\ 
%11.4 Sztuczna inteligencja\\ 
%11.5 Nauki aktuarialne\\
%11.9 Inne nauki matematyczne i informatyczne
}

%Klasyfikacja tematyczna wedlug AMS (matematyka) lub ACM (informatyka)
%TODO - dodać klasyfikację
\klasyfikacja{D. Software\\
  D.127. Blabalgorithms\\
  D.127.6. Numerical blabalysis}

%TODO - dodać słowa kluczowe]
% Słowa kluczowe:
\keywords{parameterized algorithm, kernelization, SETH, tree decomposition, cross-composition}

% Tu jest dobre miejsce na Twoje własne makra i~środowiska:

\usepackage{chngcntr}
\usepackage{amsthm}
\usepackage{amsmath}
\usepackage[]{algorithm2e}
\usepackage{enumitem}
\usepackage{datetime}
\usepackage{amssymb}
\usepackage{thmtools}
\usepackage{thm-restate}
\usepackage{hyperref}
\usepackage{cleveref}

\newdateformat{monthyeardate}{%
	\monthname[\THEMONTH], \THEYEAR}

\newtheorem{theorem}{Theorem}
\newtheorem{lemma}{Lemma}
\newtheorem{claim}{Claim}
\newtheorem{corollary}{Corollary}
\newtheorem{proposition}{Proposition}

\theoremstyle{definition}
\newtheorem{definition}{Definition}

\newcommand{\wcs}{Weighted Circuit Satisfiability}

\newenvironment{sproof}{%
	\renewcommand{\proofname}{Proof (sketch).}\proof}{\endproof}

\newcommand{\ssf}{spanning star forest}
\newcommand{\ssfp}{{\sc Spanning Star Forest}}
\newcommand{\dssfp}{{\sc Decision Spanning Star Forest}}
\newcommand{\mssfp}{{\sc Minimal Spanning Star Forest}}
\newcommand{\ssfep}{{\sc Spanning Star Forest Extension}}
\newcommand{\domset}{dominating set}
\newcommand{\domsetp}{{\sc Dominating Set}}
\newcommand{\indset}{{\sc Independent Set}}
\newcommand{\cnfsat}{{\sc CNF-SAT}}

\newcommand{\degree}[2]{\textrm{deg}_{#1}(#2)}
\newcommand{\dpt}[1]{\textrm{dp}[#1]}
\newcommand{\true}{\textrm{True}}
\newcommand{\false}{\textrm{False}}
\newcommand{\tw}{\textrm{tw}}
\newcommand{\w}[1]{\textrm{W}[#1]}
\DeclareMathOperator{\Ima}{Im}

\newcommand{\kssf}{\emph{Spanning Star Forest Problem} parameterized by the number of stars}
\newcommand{\ssfe}{\emph{Spanning Star Forest Extension Problem}}
\newcommand{\tsat}{\emph{3-SAT}}

\counterwithin{theorem}{chapter}
\counterwithin{definition}{chapter}
\counterwithin{lemma}{chapter}
\counterwithin{corollary}{chapter}
\counterwithin{claim}{chapter}
\counterwithin{proposition}{chapter}

% koniec definicji

\begin{document}
\maketitle

%tu idzie streszczenie na strone poczatkowa
%TODO - dodaj abstract
\begin{abstract}
  W~pracy przedstawiono prototypową implementację blabalizatora
  różnicowego bazującą na teorii fetorów $\sigma$-$\rho$ profesora
  Fifaka.  Wykorzystanie teorii Fifaka daje wreszcie możliwość
  efektywnego wykonania blabalizy numerycznej.  Fakt ten stanowi
  przełom technologiczny, którego konsekwencje trudno z~góry
  przewidzieć.
\end{abstract}

\tableofcontents
%\listoffigures
%\listoftables

\chapter{Introduction}

A spanning star forest of a graph is a subgraph such that it contains all the vertices and its every connected component is a tree of depth $2$. In the \ssfp{} problem, given a graph, we ask whether there exists a spanning star forest. 

The goal of this paper is to apply numerous parameterized techniques to three different variants of the problem. We start with the most basic variant, denoted \ssfp{} where we only ask whether a graph has a spanning star forest. We show a simple condition, verifiable in linear time, that is necessary and sufficient for the existence of a spanning star forest in a graph. Later, we present an algorithm that constructs a spanning star forest. The following theorem summarizes these results:

\begin{restatable}{theorem}{thmssfp}\label{thm-ssfp}
	\ssfp{} can be solved in linear time. Moreover, given a graph $G$, one can find a solution in linear time if it exists.
\end{restatable}

Afterwards, we introduce the second variant. In \mssfp{}, we look for a spanning star forest with the minimum possible number of stars. We show that \mssfp{} and \domsetp{} are essentially equivalent. That is, they are interreducible with respect to parameterized and polynomial-time reductions. Thus, we obtain with the following outcomes:

\begin{restatable}{theorem}{thmmssfpwc}\label{thm-mssfp-w2c}
	\mssfp{} is \textup{W[2]}-complete.
\end{restatable}

\begin{restatable}{theorem}{thmmssfpnpc}\label{thm-mssfp-npc}
	\mssfp{} is \textup{NP-complete}.
\end{restatable}

\noindent
Based on reductions, we show a brute force algorithm. We point out that this running time is tight, using a lower bound proved for \domsetp{} due to Pătraşcu and Williams \cite{DomSet}. 

\begin{restatable}{theorem}{thmmssfptime}\label{thm-mssfp-time}
	\mssfp{} can be solved in $\mathcal{O}^*(N^{k + o_k(1)})$, where $N$ is the number of vertices.
\end{restatable}

\begin{restatable}{theorem}{thmmssfplowerbound}\label{thm-mssfp-lowerbound}
	Unless \cnfsat{} cannot be solved in time $\mathcal{O}^*((2-\epsilon')^n)$ for some $\epsilon' > 0$, there does not exist constants $\epsilon > 0,k\geq 7$ and an algorithm solving \mssfp{} on instance with parameter equal to $k$ in time $\mathcal{O}(N^{k-\epsilon})$, where $N$ is the number of vertices of the graph.
\end{restatable}

Finally, we introduce the last variant. In \ssfep{} we are given a graph and a subset of forced edges. We ask whether there exists a spanning star forest in the graph that contains all the forced edges. In further, vertices that do not have any forced edge are called free vertices. We prove that this problem is essentially equivalent to \cnfsat{}, where the number of forced edges corresponds to the number of variables while the number of free vertices corresponds to the number of clauses. Thus, we obtain:

\begin{restatable}{theorem}{thmssfepnpc}\label{thm-ssfep-npc}
	\ssfep{} is \textup{NP}-complete.
\end{restatable}

\noindent
For the parametrization by the number of forced edges, our reductions yield the following:

\begin{restatable}{theorem}{thmssfepfetime}\label{thm-ssfep-fe-time}
	\ssfep{} parameterized by the number of forced edges can be solved in time $\mathcal{O}^*(2^{|F|})$ where $|F|$is the number of forced edges.
\end{restatable}

\begin{restatable}{theorem}{thmssfepseth}\label{thm-ssfep-seth}
	There exists an algorithm solving \cnfsat{} in time $2^{o(n)}$, where $n$ is the number of variables, if and only if there exists an algorithm solving \ssfep{} parameterized by the number of forced edges in time $2^{o(n)}$, where $n$ is the number of forced edges.
\end{restatable}

\noindent
Also, we argue that there does not exist a polynomial kernel for \ssfep{} when parameterized by the number of forced edges unless NP $\subseteq$ coNP/poly. We show two different approaches. The first one uses \cnfsat{}. In the second approach, we use the composition framework proposed by Bodlaender et al. \cite{Bodlaender}. Namely, we prove that there exists a cross-composition of \ssfep{} into itself. All in all, we obtain the following result:

\begin{restatable}{theorem}{thmssfepnokernel}\label{thm-ssfep-nokernel}
	\ssfep{} parameterized by the number of forced edges does not admit a polynomial kernel unless NP $\subseteq$ coNP/poly.
\end{restatable}

Recall that our reductions provides a link between free vertices and clauses. Therefore, when we parameterize the problem by the number of free vertices, we can immediately transform the algorithm proposed by ?? and obtain the following: %TODO ADD AUTHOR OF THE ALGORITHM%

\begin{restatable}{theorem}{thmssfepfreealg}\label{thm-ssfep-free-alg}
	\ssfep{} parameterized by the number of free vertices can be solved in ??.%TODO add time%
\end{restatable}

\noindent
Furthermore, unlike in the case of the previous parametrization, we provide an algorithm that outputs a linear kernel.

\begin{restatable}{theorem}{thmssfepkernel}\label{thm-ssfep-kernel}
	\ssfep{} parameterized by the number of free vertices admits a kernel with at most $k$ clauses and $2k$ variables.
\end{restatable}

Finally, we study the parameterization of the extension variant by the treewidth of the input graph. We propose a dynamic programming algorithm over a decomposition of a graph. Thanks to Björklund Husfeldt Kaski and Koivisto \cite{Cover product}, we prove that the algorithm works in the time stated below. Then, as previously, we show that improving the running time would give a faster algorithm for a \cnfsat{}.

\begin{restatable}{theorem}{thmssfeptwtime}\label{thm-ssfep-tw-time}
	\ssfep{} parameterized by treewidth can be solved in time $2^t\cdot poly(t)\cdot n$.
\end{restatable}

\begin{restatable}{theorem}{thmssfeptwseth}\label{thm-ssfep-tw-seth}
	Unless \cnfsat cannot be solved in time $\mathcal{O}^*((2-\epsilon')^n)$ for some $\epsilon' > 0$, there is no algorithm for \ssfep{} parameterized by treewidth that achieves running time $\mathcal{O}^*((2-\epsilon)^{t})$ for any $\epsilon > 0$, where \textup{\textrm{t}} is the treewidth of the input graph.
\end{restatable}

\chapter{Preliminaries}\label{sec2}

\section{Structures}

In a simple graph $G$ we denote by $V(G)$ and $E(G)$ the sets of vertices and of edges, respectively. 
Let $\degree{G}{v}$ denote degree of the vertex $v$ in the graph $G$ which is the number of adjacent vertices. 
An induced graph $G'$ of $G$ is a subgraph formed from a subset of vertices and all the edges between them that are present in $G$. 
For a set $X \subseteq V(G)$, by $G[X]$ we define the graph induced by vertices from $X$. 
Let $G \setminus v$ be the abbreviation for $G[V(G) \setminus \{v\}]$.
$G'$ is a \emph{subgraph} of $G$, denoted by $G' \subseteq G$, if $V(G') \subseteq V(G)$ and $E(G') \subseteq E(G)$.
A \emph{tree} $T$ is a connected graph which has exactly $|V(T)|-1$ edges. 
A \emph{spanning tree} $T$ of a graph $G$ is a connected subgraph which includes all of the vertices of $G$, with the minimum possible number of edges.
A \emph{star} $S$ is a tree of with at least $2$ vertices for which at most one vertex has a degree greater than $1$. 
A star of size at least $3$ consists of a \emph{center}, that is a vertex of the greatest degree, and \emph{rays} --- vertices of degree $1$. 
Vertices of a star of size 2 are called \emph{candidates}.
For a given graph $G$, we say that $S$ is a \emph{\ssf{}} if $V(S)=V(G)$ and every connected component of $S$ is a star.

\section{Parameterized complexity}

\emph{Parameterized complexity} is a young branch of computational complexity theory. We refer the reader to textbooks of Downey and Fellows \cite{ParComp}, Flum and Grohe \cite{ParCompThm} pr Cygan et al. \cite{ParAlg}, for an overview of the field.

We now introduce basic terminology. We begin with formally defining a parameterized problem. For the sake of clarity, all the definitions are taken from \cite{ParAlg}.

\begin{definition}\label{Parameterized problem}
	A \textit{parameterized problem} is a language $L \subseteq \ \Sigma^* \times \mathbf{N}$, where $\Sigma$ is a fixed, finite alphabet. For an instance $(x,k) \in L$, $k$ is called the \textit{parameter}.
\end{definition}

Consider the example problems:

\begin{definition}
	\indset{}: Given a graph $G$ and a positive integer $k$, decide whether there exists a set $I$ such that $|I|=k$ and $G[I]$ has no edges.
\end{definition}

\begin{definition}
	\domsetp{}: Given a graph $G$ and a positive integer $k$, decide whether there exists a set $D$ such that $|D| \leq k$ and every vertex is either in $D$ or is adjacent to one of the vertices from $D$.
\end{definition}

There are multiple different parameters for a single problem. For example, \domsetp{} can be parameterized by the sought size of dominating set $k$, or by the treewidth of the input graph. 

Now, we want to introduce different complexity classes. The first one is called FPT (fixed parameter tractable). We say that a parameterized problem is in FPT if and only if it has an FPT algorithm defined below:

\begin{definition}\label{FPT algorithm}
	For a parameterized problem $Q$, an \textit{FPT algorithm} is an algorithm $\mathcal{A}$ which, for any input $(x,k)$, decides whether $(x,k) \in Q$ in time $f(k)\cdot n^c$ where c is a constant, independent of $n,k$, and $f$ is a computable function.
\end{definition}

Another important class of parameterized problems is XP. Similarly, a problem is in XP if and only if it has an XP algorithm defined below:

\begin{definition}
	For a parameterized problem $Q$, an \textit{XP algorithm} is an algorithm $\mathcal{A}$ which, for any input $(x,k)$, decides whether $(x,k) \in Q$ in time $n^{f(k)}$ where $f$ is a computable function.
\end{definition}

Similar by polynomial-time reductions, we now introduce a \textit{parameterized reduction}, that is, a notion of transforming instances of a certain parameterized problem to instances of another one.

\begin{definition}
	Let $P,Q \subseteq \Sigma^* \times \mathbb{N}$ be two parameterized languages. A  \textit{parameterized reduction} from $P$ to $Q$ is an algorithm $\mathcal{A}$ that given $(x,k) \in P$ outputs $(x',k') \in Q$ such that the following three conditions hold:
	\begin{enumerate}
		\item $(x,k)$ is a YES-instance of $P$ if and only if $(x',k')$ is a YES-instance of $Q$.
		\item $k' \leq g(k)$ for some computable function $g$.
		\item The running time of $\mathcal{A}$ is $f(k) \cdot |x^c|$ for some computable function $f$ and constant $c$.
	\end{enumerate}
\end{definition}

Finally, we introduce the last family of complexity classes. \emph{W-hierarchy} is an ascending chain of classes : $\w{1} \subseteq \w{2} \subseteq \w{3} \subseteq...$. For the purpose of this paper, we define \w{1} as the closure of the \indset{} problem and \w{2} as the closure of the \domsetp{} problem. In other words, \indset{} parameterized by the size of independent set is  \w{1}-complete and \domsetp{} parameterized by the size of dominating set is \w{2}-complete with respect to parameterized reductions. There is a lemma that proves $\textrm{FPT} \subseteq \w{1}$ and it is conjectured that this containment is strict.

Last but not least, we introduce a \emph{kernelization algorithm} --- a way of reducing the size of input instances in polynomial time:

\begin{definition}\label{Kernel}
	A \textit{kernel} for a parameterized problem $Q$ is an algorithm $\mathcal{A}$ that, given an instance $(x,k) \in Q$, works in polynomial time and returns an equivalent instance $(x',k') \in Q$
	such that $|x'| + k' \leq g(k)$ for a computable function $g$, called the \textit{size} of the kernel.
\end{definition}

\section{Tree decomposition}

Formally, a tree decomposition of a graph $G$ is a pair $\mathcal{T} = (T, \{X_t\}_{t\in V(T)})$ where $\mathcal{T}$ is a tree whose every node $t$ is assigned a vertex subset $X_t \subseteq V(G)$, called a \emph{bag}, such that the following three conditions hold:
\begin{itemize}
	\item[(T1)] $\bigcup_{t\in V(T)}X_t = V(G)$.
	\item[(T2)] For every $vu \in E(G)$ there exists a node $t$ of $\mathcal{T}$ such that $v,u \in X_t$.
	\item[(T3)] For every $v \in V(G)$ the set $T_v = \{t \in V(T): v \in X_t\}$ induces a connected subtree of $\mathcal{T}$.
\end{itemize}

The \emph{width} of a tree decomposition $\mathcal{T} = (T,\{X_t\}_{t\in V(T)})$, denoted $\textrm{tw}(\mathcal{T})$, is equal to $\max_{t \in V(T)} |X_t| - 1$. The treewidth of a graph $G$, denoted $\textrm{tw}(G)$, is the minimum width over all tree decompositions of $G$.
\\\\
A \emph{nice tree decomposition} of a graph $G$ is a tree decomposition $(T, \{X_t\}_{t \in V(T)})$, where $T$ is rooted, such that
\begin{itemize}
	\item $X_i = \emptyset$ if $i$ is either the root or a leaf.
	\item Every non-leaf node is of one of the three following types:
	\begin{itemize}
		\item \textbf{Introduce vertex node}: a node $t$ with exactly one child $t'$ such that $X_t = X_{t'} \cup \{v\}$ for some vertex $v \notin X_{t'}$.
		\item \textbf{Introduce edge node}: a node $t$ labeled with edge $vu \in V(G)$ such that $u,v \in X_t$ with exactly one child $t'$ such that $X_t = X_{t'}$.
		\item \textbf{Forget node}: a node $t$ with exactly one child $t'$ such that $X_t = X_{t'} \setminus \{v\}$ for some vertex $v \in X_{t'}$
		\item \textbf{Join node}: a node $t$ with exactly two children $t_1$, $t_2$ such that $X_t = X_{t_1} = X_{t_2}$.
	\end{itemize}
\end{itemize}
Note that every tree decomposition can be turned into a nice one without increasing the width in time $\textrm{poly}(t) \cdot n$.

We distinguish one special case. If a tree $\mathcal{T}$ forms a path, we call it a \emph{path decomposition}. Respectively, by $\textrm{pw}(\mathcal{T})$ we denote a width of a path decomposition and by $\textrm{pw}(G)$ we denote the minimum width over all path decompositions of $G$.

\chapter{Spanning Star Forest Problem}\label{sec3}

In this chapter we examine both decision and constructive variant of \ssfp{}. We propose an algorithm working in linear time that outputs a \ssf{} or concludes that the given instance is a NO-instance.

\section{Decision variant}

In the decision variant of \ssfp{}, all that we have to do is to answer whether there exists a spanning star forest of an input graph. As it turns out, every graph that does not contain any isolated vertex has a \ssf{}.

\begin{lemma}\label{SSF lemma}
 A graph $G$ has a \ssf{} if and only if it does not contain any isolated vertices.
\end{lemma}

\begin{proof}
	If $G$ has a \ssf{} $S$, then we have that for all $v \in V(G),\ 1 \leq deg_S(v) \leq deg_G(v)$. Thus, none of the vertices is isolated.
	
	For the opposite direction, we prove the lemma by induction on $|V(G)|$. Assume $|V(G)|=2$. The statement trivially holds because a graph consisting of one edge and two vertices is a correct \ssf{}. Let $|V(G)| >2$. For the induction step, we split the proof into two parts. 
	
	Firstly, suppose that for all vertices $v \in V(G)$, it holds that $\degree{G}{v}=1$.  Clearly, $G$ is a matching. Hence, it is a \ssf{}. 
	
	Now, suppose that there exists a vertex $u$ such that $\degree{G}{u}>1$. Let $C \subseteq G$ be the connected component satisfying $u \in V(C)$. Based on the degree of $u$, we infer that $|V(C)|>2$. Let $T$ be an arbitrary spanning tree of $C$ and $v$ be one of its leaf. Observe that $T \setminus v$ is a spanning tree of $C \setminus v$. So, $C \setminus v$ does not have any isolated vertices and neither has the graph $G \setminus v$. Now, from the induction, let $S$ be a \ssf{} of the graph $G \setminus v$, $u$ be a vertex such that $uv \in E(G)$ and let $w \in N_S[u]$. Consider the two following cases:
	\begin{enumerate}
		\item Suppose $u$ is a ray in $S$. This implies that $w$ is a center and $deg_S(w) \geq 2$. Then, $S'=\big(V(S) \cup \{v\},(E(S) \cup \{uv\}) \setminus \{uw\}\big)$ is a spanning star forest for the graph $G$.
		\item Otherwise, $u$ is either a candidate or a center. Then, $S'=\big(V(S) \cup \{v\}, E(S) \cup \{uv\}\big)$ is a spanning star forest of the graph $G$. \qedhere
	\end{enumerate}
	
\end{proof}

Application of Lemma \ref{SSF lemma} yields the following result for \ssfp{}.

\begin{corollary}
	The decision variant of \ssfp{} can be solved in linear time.
\end{corollary}

\begin{proof}
	Given a graph $G = (V,E)$ the answer is YES if for all $v \in V(G)\ \degree{G}{v} \neq 0$ and NO otherwise.
\end{proof}

\section{Constructing a solution}

In the previous section, we gave an algorithm that only determines the existence of a solution. Now, we focus on constructing an arbitrary solution for a given instance. We propose an algorithm that for a graph outputs a spanning star forest in linear time if it exists. Firstly, let us introduce two claims:

\begin{claim} \label{SSF sum}
	If $C_1,C_2,...C_n$ are the connected components of a graph $G$ and $S_1,S_2,...,S_n$ are their \ssf{}s respectively, then $\bigcup\limits_{i=1}^n S_i$ is a \ssf{} for $G$.
\end{claim}

\begin{claim} \label{Spanning tree SSF}
	A connected graph $G$ has a \ssf{} if and only if its spanning tree $T$ has.
\end{claim}

The first claim can be trivially proven by the definition of a \ssf{} while the second one follows directly from Lemma \ref{SSF lemma}. Equipped with this information, we present an algorithm which solves the problem for connected graphs.

\begin{algorithm}\label{alg1}
	\KwIn{connected graph $G$ such that $|V(G)| \geq 2$}
	\KwOut{\ssf{} of $G$}
	$\textrm{spanned} \leftarrow \textrm{new Array}[|V(G)|]$\;
	$T \leftarrow$ $\textrm{SpanningTree}(G)$\;
	$S \leftarrow$ $\emptyset$\;
	\For{$v$: $\textrm{postorder}(T)$ and $v$ is not the root}{
		\If{$\textrm{not }\textrm{spanned}[v]$}{
			$u \leftarrow \textrm{parent}(T,v)$\;
			$S \leftarrow S \cup \{uv\}$\;
			$\textrm{spanned}[v] = True$\;
			$\textrm{spanned}[u] = True$\;
		}
	}
	$v \leftarrow root(T)$\;
	\If{$\textrm{not }\textrm{spanned}[v]$}{
		$u \leftarrow$ arbitrary node vertex such that $v = \textrm{parent}(T,u)$\;
		$S \leftarrow S \cup \{uv\}$\;
	}
	\Return $S$\;
	\caption{Obtaining a spanning star forest from a connected graph.}
\end{algorithm}

Firstly, the algorithm creates a spanning tree $T$. Then, it does a simple bottom-up traversal. If the current node $v$ has not been added to the solution yet, the algorithm adds the edge connecting it with parent. If the root has not been added to the solution during the for loop, we add an arbitrary edge incident to it, which finishes the algorithm.

Now we need to check that the obtained graph is a spanning star forest. There is one non-trivial operation that the algorithm does. Specifically, if the root has not  been added during the for loop, we connect the root to any existing star without checking whether it remains a correct star. Before we proceed to the lemma about the correctness of Algorithm \ref{alg1}, let us prove the following claim:
\begin{claim}\label{ssf root}
	Suppose that a connected graph $G$ is the input for Algorithm \ref{alg1}. Let $T$ be a spanning tree obtained during \textrm{SpanningTree}$(G)$ procedure and $S$ be the output graph. If $u_1 u_2,u_2 u_3 \in E(S)$, $u_2 = \textrm{parent}(T,u_1)$ and $u_3=\textrm{parent}(T,u_2)$, then $u_3$ is the root and $u_3$ has exactly one neighbor in $S$.
\end{claim}

\begin{proof}
	Observe that no two consecutive parents can be added during the for loop. Thus, edge $u_2 u_3$ must have been added in the if statement. Since $u_3 = \textrm{parent}(T,u_2)$, $u_3$ must be the root. Moreover, having known that the root becomes spanned by $S$ for the first time during the if statement, we conclude that $u_2$ is the only neighbor of $u_3$ in $S$. 
\end{proof}

\begin{lemma}\label{alg1 correctness}
	Algorithm \ref{alg1} ran on a connected graph $G$ satisfying $|V(G)| \geq 2$ outputs a spanning star forest $S$ for $G$.
\end{lemma}

\begin{proof}
	To prove the lemma, we need to show that all of the four following conditions hold after a successful execution of the algorithm:
	\begin{enumerate}
		\item $S$ does not consist of any cycle.
		\item $S$ spans $G$, i.e. $V(S) = V(G)$.
		\item $S$ does not have any isolated vertices.
		\item $S$ does not contain a path of length $3$.
	\end{enumerate}
	Let $T \subseteq G$ be a tree created during \textrm{SpanningTree}$(G)$ procedure. Trivially, $S$ does not contain a cycle because $S$ is a subgraph of $T$, which is a tree. Now, observe that the algorithm iterates over all vertices and, except for the root, pairs every vertex with its parent. The last pair, the root and its child, is added either in the for loop or in the if statement. Thus, we conclude that $S$ does not have any isolated vertices. 
	
	Finally, we prove that $S$ does not contain a path of length $3$. Note that such a path would need to contain a vertex $v$, its parent $u$ and its grandparent $w$. From the Claim \ref{ssf root} we infer that $w$ is the root and $u$ is the only neighbour of $w$ in $S$. Now, observe that if any other child of $u$ existed in that star, the last condition would still hold. So, suppose that a child of $v$ is in the same component. Contradiction, because $u$ would be the root then.
	Therefore, there does not exist a path of length $3$ and we conclude that $S$ is a spanning star forest for $G$. 
\end{proof}

Having proven the correctness of Algorithm \ref{alg1}, we proceed to the complexity analysis i.e. we prove Theorem \ref{thm-ssfp}.

\thmssfp*

\begin{proof}
	Given a graph $G$ we run Algorithm \ref{alg1} on every connected component of $G$. Then, we  merge the obtained \ssf{} in linear time. Notice that an arbitrary spanning tree of any connected component can be found in linear time. The main loop of the algorithm has $n-1$ iterations, where $n$ is the number of vertices of the component, because every vertex is processed once. Moreover, it takes constant time to finish one iteration. Thus, the total run time is linear.
\end{proof}

Thus, obtaining a \ssf{} without any limitations is easy. Both the decision and the constructive variant of the problem can be solved in linear time.

\chapter{Minimal Spanning Star Forest problem}\label{sec4}

In \mssfp{}, given a graph $G$ and a natural number $k$, the objective is to determine whether there exists a \ssf{} $S$ such that the number of connected components of $S$ is at most $k$.

It is natural to ask whether one can find a solution that minimizes the number of connected components. The problem formulated in that way resembles \domsetp{}. At first glance, one can say that a center corresponds to a dominating vertex whereas a ray corresponds to a dominated vertex. Candidates corresponds to either a dominating or a dominated vertex. However, in \domsetp{} isolated dominating vertices are allowed and some vertices can be dominated by multiple neighbors. 

To give a systematic parameterized reduction between these two problems, we need to get a better understanding of \domsetp{}.

\begin{definition}
	Given a graph $G$ and a dominating set $D$, a {\normalfont domination mapping} is a function $\mu:V(G) \setminus D \rightarrow D$ such that satisfies $(x,\mu(x)) \in E(G)$ for all $x \in V(G) \setminus D$.
\end{definition}

\begin{lemma}\label{dom mapping}
	Let $G$ be a graph without isolated vertices and let $D$ be a dominating set in $G$ of minimum size. Then, there exists a domination mapping $\mu$ such that $\mu$ is surjective.
\end{lemma}

\begin{proof}
	Let $\mu$ be a dominating mapping that maximizes $|\Ima \mu|$. If $\mu$ is surjective, then the proof is finished. Otherwise, there exists a vertex $v \in D$ such that $v \notin \Ima \mu$. Consider the following cases:
	\begin{enumerate}
		\item Suppose $N_G(v) = \emptyset$. Contradiction, $G$ has no isolated vertices. Let $u$ be any neighbor of $v$.
		\item Suppose $u \in D$. Contradiction, $D$ was assumed to be a dominating set of minimum size whereas $D \setminus \{v\}$ is a smaller dominating set.
		\item Suppose $u \notin D$ and let $w = \mu(u)$. If $|\mu^{-1}(w)|=1$, then $((D \setminus \{v,w\}) \cup u)$ is a smaller dominating set for the graph $G$. Contradiction.
		\item Finally, suppose $|\mu^{-1}(w)| > 1$. Then, the mapping:
		\begin{equation*}
			\mu'(x) = \begin{cases}
			v, & \text{if }x = u \\
			\mu(x), &\text{otherwise} \\
			\end{cases}
		\end{equation*}
		is a domination mapping that satisfies $\Ima \mu \subsetneq \Ima \mu'$. Contradiction, we assumed that $\mu$ is a dominating mapping that maximizes $|\Ima \mu|$.
	\end{enumerate}
	
	Since all the cases led to a contradiction we conclude that there exists a domination mapping $\mu$ such that $\mu$ is surjective.  
\end{proof}

In addition, we show one reduction rule that removes unnecessary vertices:

\begin{claim}\label{dom-set-rr}
	Let $(G,k)$ be an instance of \mssfp{} and $I \subseteq V(G)$ be the set of isolated vertices in $G$. Then, $(G,k)$ is a YES-instance if and only if $(G \setminus I, k-|I|)$ is a YES-instance.
\end{claim}

Claim \ref{dom-set-rr} follows by observing that every isolated vertex must be included in the dominating set. Equipped with the above information, we are ready to show the parameterized reduction:

\begin{lemma}\label{dom-ssf reduction}
	There exists a parameterized reduction that takes an instance $(G,k)$ of \domsetp{} and returns an instance $(G',k')$ of \mssfp{} such that $G' \subseteq G$ and $k' \leq k$. 
\end{lemma}

\begin{proof}
	Firstly, we modify the instance. By Claim \ref{dom-set-rr}, let $(G,',k')=(G \setminus I, k - |I|)$ be the instance without isolated vertices. If $k' < 0$ we conclude that $(G,k)$ is a NO-instance. Otherwise, we claim that $(G',k')$ is a YES-instance of \domsetp{} if and only if $(G',k')$ is a YES-instance of \mssfp{}. 
	
	Consider the backward implication. Suppose $S$ is a spanning star forest for $(G',k')$. We create the \domset{} $D$ as follows: for every star in $S$ of size $2$ pick an arbitrary candidate and for every star of size greater than $2$ pick its center. Obviously, $|D| \leq k'$ because there are at most $k'$ stars in $S$. Moreover, observe that every vertex $v \in V(G') \setminus D$ is either a ray or a candidate in $S$. Thus, there exists an edge $vu \in E(G')$ where $u \in D$.
	
	To prove the forward implication, let $D$ be a minimum size dominating set for $(G',k')$. By Lemma \ref{dom mapping}, there exists a domination mapping $\mu$ that is surjective. Now, we claim that the graph $S=(V(G'),\{x\mu(x): x \in V(G') \setminus D\})$ is a solution for the instance $(G',k')$ of \mssfp{}. Observe that $S$ has at most $k'$ components as $|D| \leq k'$ By surjectivity, there are no isolated vertices in $S$ because dominated vertices are paired with dominating ones. Moreover, $\mu$ maps vertices from $V(G') \setminus D$ to $D$. Thus, we obtain that for all $v \in V(G') \setminus D$, $\degree{S}{v}=1$ and there does not exist an edge in $S$ connecting two dominating vertices. Thus, $S$ is a spanning star forest.
\end{proof}

The problems look so similar that one could ask whether there exists a reverse parameterized reduction. Indeed, this is true and the following lemma formally proves it:

\begin{lemma}\label{ssf-dom reduction}
	There exists a parameterized reduction that takes an instance $(G,k)$ of \mssfp{} and returns an instance $(G,k')$ of \domsetp{} such that $k' \leq k$. 
\end{lemma}

\begin{proof}
	Let $(G,k)$ be an instance of \mssfp{}. If $G$ contains an isolated vertex, then return $(G,0)$. Otherwise, return $(G,k)$. Now, we claim that $(G,k)$ is a YES-instance of \mssfp{} if and only if $(G,k')$ is a YES-instance of \domset{} where $k'=0$ or $k'=k$.
	
	Consider the forward implication. Let $S$ be a spanning star forest of at most $k$ stars for the graph $G$. Then, by Lemma \ref{SSF lemma}, $G$ does not have any isolated vertices and $k'=k$. Observe that $G$ does not change during the reduction. We create a dominating set $D$ as follows: for every star of size $2$ pick an arbitrary candidate and for every star of size at least $3$ pick center. Obviously, $|D| \leq k$ because $S$ contains at most $k$ stars. So, suppose that there exists $v \in V(G) \setminus D$ that is not dominated. However, $S$ spans $G$ which means that $v$ is in one of the stars. Therefore, not only $v$ is either a ray or a candidate in $S$, but also there exists an edge $vu \in E(S)$ such that $u$ is either a center or a candidate. By the definition of $D$, $u \in D$. Contradiction because $u$ dominates $v$.
	
	For the backward implication, let $D$ be a minimum size dominating set for $(G,k')$. Note that $G$ has no isolated vertices and $k'=k$. By Lemma \ref{dom mapping}, there exists a domination mapping $\mu$ such that $\mu$ is surjective. Now, we claim that the graph $S$ = $(V(G), \{x\mu(x): x \in V(G) \setminus D\})$ is a spanning star forest. $S$ spans $G$ as it contains all the vertices from $G$. Moreover, there are no isolated vertices because for every vertex $v \in V(G) \setminus D$ there exists exactly one vertex $u \in D$ such that $vu \in E(S)$ and for every vertex $u \in D$ there exists at least one vertex $v \in V(G) \setminus D$ such that $vu \in E(S)$. From the previous sentence we also infer that the mapping forces every connected component to be a star, which concludes the proof.
\end{proof}

Provided that there exist reductions from \domsetp{} to \mssfp{} and from \mssfp{} to \domsetp{} we are ready to prove the main results.

\thmmssfpwc*

\begin{proof}
	Recall, that \domsetp{} is a W[2]-complete problem. Note that by Lemma \ref{dom-ssf reduction} and Lemma \ref{ssf-dom reduction} the problems are equivalent with respect to parameterized reductions. Thus, \mssfp{} is W[2]-complete.
\end{proof}

Moreover, note that the parameterized reductions stated in the previous lemmas can be considered as polynomial reductions. Hence, we get:

\thmmssfpnpc*

\begin{proof}
	\domsetp{} is an NP-complete problem. As observed in the previous proof, \domsetp{} and \mssfp{} are equivalent with respect to polynomial reductions. Therefore, we conclude that \mssfp{} is  also NP-complete.
\end{proof}

Interreducibility is a useful tool. Especially, if one of the problems has been deeply studied in the past. As an example, we show how to transfer a lower bound for the running time from \domsetp{} to \mssfp{}. Firstly, we prove the existence of an algorithm that works in time stated in Theorem \ref{thm-mssfp-time}:

\thmmssfptime*

\begin{proof}
	Let $(G,k)$ be an instance of \mssfp{}. We apply the reduction from Lemma \ref{ssf-dom reduction} and obtain an instance $(G,k')$ of \domsetp{}. It is known that there exists an algorithm solving \domsetp{} in time $\mathcal{O}(N^{k + o_k(1)})$ \cite{DomSetAlg}. This concludes the proof.
\end{proof}

Now, consider the following theorem proven by Pătraşcu and Williams \cite{DomSet}:

\begin{theorem}\label{domset-seth}
	Unless \cnfsat{} cannot be solved in time $\mathcal{O}^*((2-\epsilon')^n)$ for some $\epsilon' > 0$, there does not exist constants $\epsilon > 0, k \geq 7$ and an algorithm solving \domsetp{} on instance with parameter equal to $k$ that run in time $\mathcal{O}(N^{k-\epsilon})$, where $N$ is the number of vertices of the input graph.
\end{theorem}

Armed with the theorem and reductions, we are ready to prove the last result for \mssfp{}:

\thmmssfplowerbound*

\begin{proof}
	Assume \cnfsat{} cannot be solved in the stated time. However, suppose there exist a constant $\epsilon > 0$ and an algorithm $\mathcal{A}$ that solves \mssfp{} instance in time $\mathcal{O}(N^{k-\epsilon})$, where $k \geq 7$ and $\epsilon > 0$ are fixed. Let $(G,k)$ be an instance of \domsetp{}. We show an algorithm that contradicts Theorem \ref{domset-seth}. Firstly, we apply the reduction stated in Lemma \ref{dom-ssf reduction}. We obtain an instance $(G',k')$ of \mssfp{} such that $G' \subseteq G$ and $k' \leq k$. Then, we can apply algorithm $\mathcal{A}$ to obtain the answer in time $\mathcal{O}(N^{k'-\epsilon})$, where $N=|V(G')| \leq |V(G)|$ and $k'\leq k$. Contradiction.
\end{proof}

\chapter{Spanning Star Forest Extension}\label{sec5}

In this chapter, we study a significantly different variant of the \ssfp{} problem. Let $G$ be a graph and $F \subseteq E(G)$ be a set of \emph{forced edges}. In the \ssfep{} we ask whether there exists a \ssf{} $S$ such that $F \subseteq E(S)$.

In further, we denote by $F$ a set of \emph{forced edges}. Vertices that have exactly one forced edge are called \emph{forced candidates}. Similarly, if a subset of $F$ forms a \emph{forced star} of size greater than $2$, then we call its particles a \emph{forced center} and \emph{forced rays} consequently. We denote by $F_R$ a set of all forced rays and by $F_C$ a set of all forced centers. Vertices that does not belong to $V(F)$ are called \emph{free vertices} and they are denoted by $U$. 

We consider three different parameters for this problem: the number of forced edges, the number of free vertices and the treewidth. 

\section{Instance normalization} 

Notice that this time we do not have any restriction on the number of connected components. The hardness of the problem lies in choosing which of the forced candidates should become centers and which should become forced rays. 

In this section we propose a definition of a \emph{normalized instance} --- an instance which satisfies a set of conditions described below. Note that an \emph{induced matching} $M$ in a graph $G$ is a set of disjoint edges such that there are no edges outside of $M$ with both endpoints in $V(M)$.

\begin{definition}\label{norm-ssfe}
	A pair $(G,F)$ is normalized if the following conditions hold:
	\begin{enumerate}
		\item $G$ does not have isolated vertices.
		\item $F$ is an induced matching.
		\item For every $u \in U$, all neighbors of $u$ are in $V(F)$.
	\end{enumerate}
\end{definition}

Surprisingly, every instance of \ssfep{} can be either normalized or discarded as a NO-instance. This is explained in the following lemma:

\begin{lemma}
	There is an algorithm working in polynomial time that takes an instance $(G,F)$ of \ssfep{} and either conclude that the instance is a NO-instance or outputs an equivalent normalized instance $(G',F')$ satisfying $G' \subseteq G$ and $F' \subseteq F$ and $U' \subseteq U$, where $U'$ is the set of free vertices in $G'$.
\end{lemma}

\begin{proof}
	We begin by showing cases for which we conclude that $(G,F)$ is a NO-instance:
	\begin{enumerate}[leftmargin=*,label=\textbf{Reduction \arabic{enumi}},labelindent=0pt]
		\item If graph $G$ contains an isolated vertex, then $(G,F)$ is a NO-instance.
	\end{enumerate}
	
	Obviously, an isolated vertex cannot be a star by Lemma \ref{SSF lemma}. Now, observe, that in the extension variant, there exists a set of edges that must be added to the solution. Therefore, we can instantly conclude that an instance is a NO-instance if $F$ forms a forbidden subgraph.
	
	\begin{enumerate}[leftmargin=*,label=\textbf{Reduction \arabic{enumi}},labelindent=0pt,resume]
		\item If $F$ contains a path or a cycle of length at least $3$, then $(G,F)$ is a NO-instance.
	\end{enumerate}
	
	Now, let us show three reduction rules. After their exhaustive application, the set of forced edges becomes an induced matching. Firstly, we remove free edges between forced vertices:	

	\begin{enumerate}[leftmargin=*,label=\textbf{Reduction \arabic{enumi}},labelindent=0pt,resume]
		\item Remove the set of edges $\{vu: v,u \in V(F),\ vu \in E(G) \setminus F\}$.
	\end{enumerate}
	Clearly, if such an edge was included in a solution $S$, then the solution would contain a path or a cycle of length $3$. Thus, the operation is safe. 
	
	Now suppose that a subset of forced edges forms a star of size at least $3$. Then, the forced center is already determined. Hence, we can remove from the instance all the free edges that have at least one end in a forced ray. 
	
	\begin{enumerate}[leftmargin=*,label=\textbf{Reduction \arabic{enumi}},resume,wide, labelwidth=!, labelindent=0pt]
		\item Remove the set of edges $\{uv: v \in F_R,\ uv \in E(G) \setminus F\}$
	\end{enumerate}
	We claim that the operation is safe. To prove it, suppose contrary. Let $u \in V(G)$, $v \in F_R$ and $uv \in E(G) \setminus F$. Now, suppose that there exists a solution $S$ such that $uv \in E(S)$. However, $v$ is a forced ray. So there exists a vertex $c \in F_C$ and $v' \in F_R$, such that $v \neq v'$ and $vc,cv' \in F$. Moreover, $vc,cv' \in E(S)$. If $u=v'$, then $S$ contains a cycle of length $3$. Otherwise, $uv,vc,cv$ form a path of length $3$. Thus, $S$ is not a spanning star forest.

	Observe that after exhaustive application of the above rules, every $v \in F_R$ satisfies $\degree{G}{v}=1$. Every forced ray is connected to its forced center only. Hence, for every forced star of size greater than $2$ we can remove all forced rays except for one.


	\begin{enumerate}[leftmargin=*,label=\textbf{Reduction \arabic{enumi}},resume,wide, labelwidth=!, labelindent=0pt]
		\item Suppose that $(G,F)$ is the output graph after exhaustive application of the previous reductions. For every forced star with more than $2$ vertices, remove all the forced rays except for one.
	\end{enumerate}

	We prove the safeness now. Let $(G',F')$ be the output instance after application of Reduction 5. We claim that $(G',F')$ has a \ssf{} if and only if $(G,F)$ has a \ssf{}. 
	
	For the forward implication, let $S$ be a solution for $(G',F')$. Let $S_C$ be a set of centers in $S$. Recall that by $F_C$ we denote the set of forced centers in $G$ and by $F_R$ we denote the set of forced rays. We claim that $S_C \subseteq F_C$. Indeed, observe that Reduction 4 removes all the free edges that has at least one end in a forced ray. So it follows that for all $v \in V(G') \cap F_R$, $\degree{G'}{v}=1$ and $\degree{G'}{v} \geq \degree{S}{v} \geq 1$. Therefore, we conclude that $S \cup (F \setminus F')$ is a spanning star forest for $G$ as we always connect a removed vertex to a center or a candidate.
	
	Conversely, let $S$ be a spanning star forest for $(G,F)$. Now, let $S' \subseteq S$ be a subgraph restricted to vertices of $G'$. Observe that every forced star in $G$ remains a forced star in $G'$ as $F'$ is a matching. Moreover, removed vertices do not have any edges to free vertices in $G$ by Reduction 4. Thus, $S'$ is a spanning star forest for the graph $G'$.

	Let us summarize the work and describe how the instance looks like after exhaustive application of Reductions 1-5:

	\begin{claim}
		Given an instance $(G,F)$, if Reductions 1-2 do not yield that a graph is a NO-instance, then exhaustive application of Reductions 3-5 outputs $(G',F')$ such that $F'$ is an induced matching in $G'$.
	\end{claim}
	
	\begin{proof}
		We prove the claim by contradiction. Firstly, suppose that $F$ is not a matching. Hence, there exists a connected component with more than $2$ vertices. It must be a star because Reduction 2 does not yield that $(G,F)$ is a NO-instance. Contradiction, we did not apply exhaustively Reduction 5 to decrease the size of each forced star. Now, suppose that $F$ is not an induced matching. Hence, there exist vertices $v,u \in V(F)$ such that $vu \in E(G) \setminus F$. Contradiction, Reduction 3 has not been applied exhaustively.
	\end{proof}

	Now, let us focus on the second part of the graph i.e. free vertices. As we have already seen, by Lemma \ref{SSF lemma}, there exists a spanning star forest if and only if there are no isolated vertices. Let $V_P = \{u: \text{there exists } v\in U \text{ such that }uv \in E(G)\}$ and $V_{NP} = V(G) \setminus V_P$. Finally, $G_{NP} = G[V_{NP}]$ and $G_P = G[V_P]$. Now, we claim that:

	\begin{claim}\label{GP partition}
		$G_P$ has a spanning star forest.
	\end{claim}
	
	\begin{proof}
		By the definition of $G_P$, Every vertex has at least one neighbor in $G_P$. Hence, by Lemma \ref{SSF lemma} $G_P$ has a spanning star forest.
	\end{proof}

	Observe that during partitioning $G$ into $G_P$ and $G_{NP}$ we lose the information about some edges. Specifically, let $L= \{vu: vu \in E(G), v \in V(G_{NP}),\ u \in V(G_P)\}$ be the set. Note that vertices from $G_{NP}$ that are incident to $L$ are the forced vertices. Additionally, both forced vertices and vertices from $G_P$ are already satisfied i.e. we can always span them by a star forest. Therefore, we can formally state the following claim:
	
	\begin{claim}\label{GNP partition}
		Let $(G,F)$ be an instance of \ssfep{} after exhaustive application of Reductions 1-5. Then $(G,F)$ has a solution if and only if $(G_{NP},F)$ has one.
	\end{claim}

	\begin{proof}
		For the backward implication, suppose $S$ is a solution for $(G_{NP},F)$. We  partition $G$ into $G_P$ and $G_{NP}$. By Lemma \ref{GP partition}, let $S'$ be a solution for $G_P$. Then, $S \cup S'$ is a \ssf{} for $G$.
		
		Now, consider the forward implication. Let $S$ be a solution for $(G, F)$ and let $S' = S \cap E(G_{NP})$, be a subgraph restricted to the edges of $G_{NP}$ only. Observe that $S'$ is a star forest. If $S'$ is a spanning star forest for $G_{NP}$ then we conclude. Otherwise, there exists a vertex $v$ such that $v \in V(G_{NP}) \setminus V(S')$. Note that $v$ is a free vertex because all the forced vertices are spanned by forced edges from $F$. However, by definition of $G_P$, $v$ has no neighbors outside $G_{NP}$. Hence, $v \notin V(S)$ and $S$ is not a spanning star forest for $G$. Contradiction.
	\end{proof}

	Recall that free vertices of $G_{NP}$ are not adjacent to other free vertices. Thus, the instance $(G_{NP},F)$ satisfies the last condition of Definition \ref{norm-ssfe}.

	To conclude, let $(G,F)$ be an input graph. If Reduction 1 or Reduction 2 applies to $(G,F)$ then it is a NO-instance. Otherwise, we apply Reductions 3-5 exhaustively, in order, and obtain $(G',F')$ . Finally, we partition $G'$ into $G'_{NP}$ and $G'$. As an output we return an instance $(G'_{NP},F')$ that follows $G'_{NP} \subseteq G$, $F' \subseteq F$ and $U' \subseteq U$, where $U'$ is the set of free vertices in $G'_{NP}$. \qedhere

\end{proof}

Finally, we want to point out an advantage of a normalized instance of \ssfep{} over an arbitrary one. Consider the following claim:

\begin{claim}\label{span-lemma}
	Let $(G,F)$ be a normalized instance of \ssfep{}. Then, $(G,F)$ is a YES-instance if and only if there exists an independent set $C \subseteq V(F)$ such that $U \subseteq N(C)$.
\end{claim}

\begin{proof}
	For the forward implication, let $S$ be a \ssf{} for $(G,F)$. We claim that $C = N_S(U)$ satisfies the required condition. Clearly, $C \subseteq V(F)$ as free vertices have edges to forced vertices only. Additionally, for every forced edge $vu \in F$ either $v$ or $u$ does not have edge to free vertices in $S$. Thus, $C$ is an independent set. 
	
	For the backward implication, observe that $F$ is an induced matching. If $U \subseteq N[C]$, then for every $v \in U$ there exists a vertex $u \in C$ such that $vu \in E(G)$. Thus, we can add every free vertex to one of the existing stars. Let $S \subseteq G$ be a subgraph that takes all the forced edges and, for every free vertex, takes exactly one edge to a forced vertex from $C$. Such edges exist because $U \subseteq N(C)$. Observe that in $S$, for every $uv \in F$, the degree of at most one vertex is greater than $1$ because $C$ is an independent set. Thus, $S$ is spanning star forest for $(G,F)$.
\end{proof}



\section{NP-completeness}

In this section we present a parameterized reductions from \ssfep{} to \cnfsat{}. As it turns out, the number of forced edges corresponds to the number of variables and the number of free vertices corresponds to the number of clauses.

\begin{lemma}\label{ssfep reduction}
	There exists a polynomial time reduction that takes an instance $\phi$ of \cnfsat{}, say with $n$ variables and $m$ clauses, and returns an equivalent normalized instance $(G,F)$ of \ssfep{} such that $|V(G)|=2n+m$, $|F|=n$ and $|U|=m$.
\end{lemma}

\begin{proof}
	Firstly, we present the reduction. Suppose $\phi$ is the input instance of \cnfsat{}. As $\phi$ is a CNF formula, let $\textrm{Clauses}=\{C_1,...,C_m\}$ be the set of clauses and let $\textrm{Variables}=\{x_1,...,x_n\}$ be the set of variables in $\phi$. For every clause $C_i$ we introduce a vertex $v[C_i]$ and for every variable $x_i$ we introduce two vertices $v[x_i],v[\neg x_i]$ and a forced edge $v[x_i]v[\neg x_i]$. Observe that the graph consists of $2n+m$ vertices and $n$ forced edges. Now, for every occurrence of a literal $l_i$ in a clause $C_j$ we introduce a free edge $v[C_j]v[l_i]$. Finally, we say that $(G,F)$, the graph that we described, has a spanning star forest that extends $F$ if and only if there exists an assignment satisfying the formula $\phi$.
	
	Before we begin the proof, observe that $(G,F)$ is a normalized instance. Firstly, there are no isolated vertices because every clause has at least one literal and variables corresponds to a forced edge. Secondly, $F$ is an induced matching because vertices introduced for literals (forced vertices) are connected to vertices introduced for clauses (free vertices) only. And finally, no edges were introduced between vertices introduced for clauses (free vertices).
	
	For the forward implication, let $S$ be a solution for $(G,F)$. We create an evaluation $\sigma$ as follows:
	
	\begin{equation*}
		\sigma(x_i) = 
		\begin{cases}
			\textrm{True}\text{, if $\degree{S}{v[x_i]} > 1$} \\
			\textrm{False}\text{, otherwise}
		\end{cases}
	\end{equation*}
	We claim that the evaluation satisfies $\phi$. Fix an arbitrary clause $C_i$. Now, observe that there exists a literal $l_j \in C_i$ such that $v[l_j]v[C_i] \in E(S)$. Then, $\degree{S}{l_j}>1$. By the definition of $\sigma$, $\sigma(l_j)=\true$, and therefore $\sigma(C_i)=\true$. We conclude that $\sigma(\phi)=\true$ as we proved that an arbitrary clause in the formula is satisfied.
	
	To prove the backward implication, assume there exists an evaluation $\sigma$ of variables that satisfies the formula. If $\sigma(\phi)=\textrm{True}$, then for every $C_i \in \textrm{Clauses}$, $\sigma(C_i)=\textrm{True}$. Moreover, for every $C_i \in \textrm{Clauses}$ there exists a literal $l_i \in C_i$ such that $\sigma(l_i)=1$. Now, let $L = \{v[l]: \sigma(l)=1\}$. Clearly, $\{v[C_i]: C_i \in \textrm{Clauses}\} \subseteq N_G(L)$ because $\sigma$ satisfies the formula $\phi$. Moreover $L$ is an independent set in $G$ because for every forced edge $v[x_i]v[\neg x_i] \in F$ either $\sigma(x_i)=1$ or $\sigma(\neg x_i)=1$. Hence, by Lemma \ref{span-lemma}, there exists a spanning star forest for $(G,F)$.
\end{proof}

There is one more observation that we want to point out in this section. Since a CNF-formula is trivially encoded as a spanning star forest extension instance, one can ask if the problems are interreducible. Indeed, it is true and we present the backward reduction.

\begin{lemma}\label{cnfsat reduction}
	There exists a polynomial time reduction that takes an instance $(G,F)$, such that $(G,F)$ has $n$ forced edges and $m$ free vertices, and returns a formula $\phi$ of \cnfsat{} such that $\phi$ has at most $n$ variables and at most $m$ clauses.
\end{lemma}

\begin{proof}
	Firstly, we apply Lemma \ref{norm-ssfe} to normalize the instance. If it yields a no instance, we return a formula $(x \land \neg x)$. Otherwise, let $(G',F')$ be the output of the exhaustive application of the reduction rules. Observe that $G' \subseteq G$ and $F' \subseteq F$. Now, we proceed to a formula construction. For every forced edge $vu$ we introduce a variable $x_{vu}$ and we arbitrarily label its ends as $x_{vu}$ and $\neg x_{vu}$. Now, for every free vertex $w$ we introduce a clause $C_w$. Moreover, every clause $C_w$ consists of a disjunction of literals $\textrm{labels}(N_G(w))$. Finally, we claim that $(G,F)$ has a spanning star forest if and only if the formula $\phi$, described above, is satisfiable.
	
	Observe that the instances are equivalent to the instances described in Lemma \ref{ssfep reduction}. One can follow the reasoning as in the previous reduction.
\end{proof}

Finally, we can prove the main theorem of the section:

\thmssfepnpc*

\begin{proof}
	We apply the reduction described in Lemma \ref{ssfep reduction} from an NP-complete problem, that is, \cnfsat{}.
\end{proof}

\section{Parametrization by the number of forced edges}

In this section, in addition to an instance $(G,F)$ we receive a parameter $k$ which is equal to the number of forced edges. We show two major results: \ssfep{} parameterized by the number of forced edges does not admit a kernel of polynomial size and a lower bound under Strong Exponential Hypothesis.

\subsection{Lower bound for a run time}

Previously in this chapter, we proved that \ssfep{} is NP-complete. We showed that the problem is NP-complete, it does not admit a polynomial kernel and we stated reduction rules to simplify instances. In this subsection, we show a simple routine that solves \ssfep{} parameterized by the number of forced edges. Furthermore, we prove that there does not exist a faster algorithm unless \cnfsat{} cannot be solved in time $\mathcal{O}^*((2-\epsilon)^n)$, for $\epsilon>0$.

\begin{algorithm}\label{alg2}
	\KwData{normalized instance $(G,F)$}
	\KwResult{\ssf{} of $G$ extending $F$}
	$Centers \leftarrow \{C: C \subseteq V(F) \text{, }|C|=|F| \text{ and } \forall u,v \in C,\ vu \notin F\}$\;
	\For{$C \in Centers$ }{
		\If{$U \subseteq G(C)$}{
			\Return YES-instance\;
		}
	}
	\Return NO-instance\;
	\caption{Extending a spanning star forest from a normalized graph.}
\end{algorithm}

Consider the following Algorithm \ref{alg2}. It simply iterates over all maximal independent sets of forced candidates. If a set spans all the vertices, then it means that the set of forced edges can be extended to a spanning star forest. Otherwise, if none of the sets satisfies the condition, then the input is a NO-instance. Now, see the following lemma:

\begin{lemma}\label{alg2-correctness}
	Given a normalized instance $(G,F)$ parameterized by $|F|$, Algorithm \ref{alg2} outputs the answer whether $(G,F)$ has a spanning star forest.
\end{lemma}

\begin{proof}
	Lemma \ref{span-lemma} proves the correctness of the algorithm.
\end{proof}

With Lemma \ref{alg2-correctness}, we can proceed to the next theorem stated in the introduction:

\thmssfepfetime*

\begin{proof}
	Firstly, we normalize the input instance by Lemma \ref{norm-ssfe} which takes a polynomial time. We either conclude that an instance is a NO-instance or obtain $(G,F)$. Then, we apply Algorithm \ref{alg2}. It does at most $2^{|F|}$ iterations because this is the number of different maximal independent sets in an induced matching. Every iteration takes polynomial time to process the set. Hence, we conclude that the algorithm works in time $\mathcal{O}^*(2^{|F|})$.
\end{proof}

Note that the described algorithm is a simple brute force. We do not optimize the search. Moreover, there is no need to fight for a better complexity unless SETH fails. The following theorem proves it:

\thmssfepseth*

\begin{proof}
	Suppose $\mathcal{A}$ is an algorithm that solves \cnfsat{} in time $2^{o(n)}$, where $n$ is the number of variables. Now, we show an algorithm solving \ssfep{} parameterized by the number of forced edges. Let $(G,F)$ be an arbitrary instance of \ssfep{} with a parameter equal to $|F|$. We apply the reduction from Lemma \ref{ssfep reduction} and obtain a formula $\phi$, that has exactly $|F|$ variables. Then, we apply algorithm $\mathcal{A}$ to obtain the result. Observe that the reduction works in polynomial time. Thus, the above algorithm works in time $2^{o(n)}$.
	
	For the converse implication, assume $\mathcal{A}$ is an algorithm that solves \ssfep{} parameterized by the number of forced edges in time $2^{o(n)}$, where $n$ is the number of forced edges. We show an algorithm for \cnfsat{} problem now. Let $\phi$ be an arbitrary CNF-formula with $n$ variables. We apply the reduction from Lemma \ref{cnfsat reduction} and obtain a normalized instance $(G,F)$ of \ssfep{}, such that $|F|=n$. Now, we run the algorithm $\mathcal{A}$ on $(G,F)$, and hence we get the answer. The described algorithm works in time $2^{o(n)}$.
\end{proof}

\subsection{Lower bound for a kernel}

In this section, we prove that \ssfep{} parameterized by the number of forced edges does not admit a polynomial kernel unless some classes collapse. To achieve this, we show two different approaches. Firstly, we show a proof based on parameterized reductions stated in previous subsection. Consider the following lemma:

\begin{lemma}\label{kernel}
	\cnfsat{} parameterized by the number of variables has a polynomial kernel if and only if \ssfep{} parameterized by the number of vertices has one.
\end{lemma}

\begin{proof}
	For the forward implication, let $\mathcal{A}$ be such kernelization algorithm. Now, we show a kernelization algorithm for \ssfep{} problem. Let $(G,F)$ be an arbitrary instance. We apply the reduction from Lemma \ref{cnfsat reduction} and obtain a formula $\phi$ of $|U|$ clauses and $|F|$ variables. Now, we apply the algorithm $\mathcal{A}$ and obtain a formula $\phi'$ such that $|\phi'| \leq \textrm{poly}(|F|)$. Finally, we create an instance $(G',F')$ by the reduction from \ref{ssfep reduction}. Note that now, $|G'|+|F'| \leq \textrm{poly}(|F|)$ as the second reduction does not change the size.
	
	Observe that for the converse implication, one can use the same reasoning.
\end{proof}

Now, we state the following result of Fortnow and Santhanam \cite{CNFSAT}:

\begin{theorem}\label{Fortnow}
	\cnfsat{} is not polynomially kernelizable unless NP $\subseteq$ coNP/poly.
\end{theorem}

Immediately, we get:

\thmssfepnokernel*

\begin{proof}
	The theorem follows directly from Lemma \ref{kernel} and Theorem \ref{Fortnow}.
\end{proof}

The second approach that we show is completely different. We show a notion of \textit{cross-composition}. It is a framework for proving kernelization lower bounds. A technique, firstly introduced in 2008 by Bodleander et al. \cite{Bodlaender} has significantly increased the interest in kernelization. We know introduce the schema. The following definitions and corollary are taken from Parameterized Complexity \cite{ParAlg} book.

\begin{definition}\label{polynomial equivalence relation}
	An equivalence relation $\mathcal{R}$ on $\Sigma^*$ is called a \textit{polynomial equivalence relation} if the following conditions hold:
	\begin{enumerate}
		\item There exists an algorithm $\mathcal{A}$ such that given given $x,y \in \Sigma^*$ decides whether $x \equiv_{\mathcal{R}} y$ in time $p(|x|+|y|)$ for a polynomial $p$..
		\item Relation $\mathcal{R}$ restricted to the set $\Sigma^{\leq n}$ has at most polynomially many equivalence classes.
	\end{enumerate}
\end{definition}

\begin{definition}\label{cross-composition}
	Let $L \subseteq \Sigma^*$ be a language, $\mathcal{R}$ be an equivalence relation $Q \subseteq \Sigma^* \times \mathbb{N}$ be a parameterized problem. A \textit{cross-composition} of a language $L$ into $Q$ is an algorithm $\mathcal{A}$ that given an input $x_1,...,x_t \in L$, equivalent with respect to $\mathcal{R}$, outputs an instance $(x,k') \in \Sigma^* \times \mathbb{N}$ such that:
	\begin{enumerate}
		\item $k \leq p(\max\limits_{1 \leq i \leq t} |x_i| + log(t))$ for a polynomial $p$.
		\item $(x,k') \in Q$ if and only if there exists an index $i$ such that $x_i \in L$.
	\end{enumerate}
\end{definition}

\begin{corollary}\label{nokernel}
	If an NP-hard language $L$ cross-composes into the parameterized language $Q$, then $Q$ does not admit a polynomial kernel unless \textit{NP$\subseteq$ coNP/poly}.
\end{corollary}

We prove nonexistence of a polynomial kernel by a cross-composition from \ssfep{} into itself. Observe that \ssfep{} is NP-complete by Theorem \ref{thm-ssfep-npc}. In the proof, we use an \emph{instance selector}, a pattern commonly applied to solve a composition. Intuitively, we need to come up with a gadget that satisfies all instances but one. Therefore, we require that at least one of the packed instances has a solution.

\begin{lemma}\label{cross-ssfe}
	There exists a cross-composition from \ssfep{} into itself, parameterized by the number of forced edges.
\end{lemma}

\begin{proof}
	Firstly, we define a relation $\mathcal{R}$. Assume that all malformed graphs are considered as equivalent. Moreover, we say that $(G_1,F_1) \equiv_\mathcal{R} (G_2,F_2)$ if and only if graphs induces by forced vertices are isomorphic. Observe that the relation implies that the graphs have the same amount of forced edges.
	
	Now, let $(G_1,F_1),\cdots,(G_t,F_t)$ be the input instances. We normalize them by applying Lemma \ref{norm-ssfe}. If all of the instances are NO-instances, we return an isolated vertex. Otherwise, let $(H_0,F'_0,\cdot,H_{p-1},F'_{p-1})$ be the normalized instances. Because we operate on binary representation of indices, we duplicated some instances so that $p=2^s$ for some integer $s>0$. Observe that the step at most doubles the number of instances.
	
	We now proceed to the construction of the output instance. Firstly, for every input instance we label forced edges with $f_1,...,f_n$. Then, let $(G,F)$ be a sum of graphs $H_0,...,H_{p-1}$ such that the forced edges with the same labels are unified. Thus, we have that $|F|=n$. Now, we introduce $2s$ forced vertices $v_\alpha^\beta$ where $\beta \in \{0,1\}$ and $0 \leq \alpha \leq s-1$, and $s$ forced edges $v_\alpha^0v_\alpha^1$. In addition, for every forced edge $v_\alpha^0v_\alpha^1$ we introduce a free vertex $v_\alpha$ and edges $v_\alpha^0v_\alpha,v_\alpha^1v_\alpha$. Observe that now $|F|\leq n + \log(t)$. Recall that by $U_i$ we denote a set of free vertices of the graph $G_i$. For every index $i$, where $0 \leq i \leq p-1$, we do the following: let $i=b_0b_1...b_{s-1}$ be a bit representation. If necessary, we add leading zeros so that every value is represented by $s$ bits. For every vertex $v \in U_i$ we introduce a set of edges $\{vv^{1-b_\alpha}_\alpha: 0 \leq \alpha \leq s-1\}$.
	
	We output the modified instance $(G,F)$. Observe that $(G,F)$ is also a normalized instance. The first condition of the cross-composition is satisfied as $|F| \leq n + \log(p)$. So, to finish the proof, we need to show that one of the instances $(G_i,F_i)$ is a YES-instance if and only if $(G,F)$ is a YES-instance.
	
	For the forward implication, let $S$ be a spanning star forest for $(G_i,F_i)$. Let $C_S = N_S(U_i)$ be a set of centers in $S$. We construct a solution for $(G,F)$ as follows. Let $i=b_0...b_{s-1}$ be a bit representation with leading zeros if necessary. Let $C_G = \{v_{\alpha}^{b_\alpha}: 0 \leq \alpha \leq s-1\}$. By the definition of $G$, we get that $N(C_S \cup C_G) = U$. Thus, by Lemma \ref{span-lemma}, $(G,F)$ has a spanning star forest.
	
	Conversely, let $S$ be a solution for $(G,F)$. We define $C_G = N(\{v_\alpha: 0 \leq \alpha \leq s-1\})$. Superscripts of vertices $v^{\alpha_0}_0,...,v^{\alpha_{s-1}}_{s-1}$ create a value $i$. Now, observe that $N(U_i) \subseteq V(F)$ because the superscripts of vertices $v^{\alpha_j}_j$, for $0 \leq j \leq s-1$, match bit representation of $i$. Thus, $(G_i,F_i)$ is a yes instance.
\end{proof}

\thmssfepnokernel*

\begin{proof}
	By Corollary \ref{nokernel} and Lemma \ref{cross-ssfe} we obtain that \ssfep{} parameterized by $|F|$ does not admit a polynomial size kernel unless \textup{NP $\subseteq$ coNP/poly}.
\end{proof}

\section{Parametrization by the number of free vertices}

In this section, we parameterize  \ssfep{} by the number of free edges. We begin with the algorithm that solves the problem in time ??.%TODO%. 
Then, unlike in the first variant, we present a kernelization algorithm. We show, that the reduced instance has at most $3k$ vertices where $k$ is the parameter.

\subsection{Algorithm}
We start with stating a theorem for the \cnfsat{} problem parameterized by the number of clauses proven by AUTHORS. %TODO

\begin{theorem}\label{cnfsatmtime}
	TODO
\end{theorem}

Now, observe that \cnfsat{} parameterized by the number of clauses and \ssfep{} parameterized by the number of free vertices are interreducible with respect to polynomial reductions. Thus, we are ready to show an algorithm.

\thmssfepfetime*

\begin{proof}
	Let $(G,F)$ be an arbitrary instance of \ssfep{} parameterized by the number of free vertices. Firstly, we apply Lemma \ref{norm-ssfe}. If the lemma yields that the instance is a NO-instance, then we conclude. Otherwise, let $(G',F')$ be the output graph of normalization lemma. We apply Lemma \ref{cnfsat reduction} and obtain a formula $\phi$. Observe that $\phi$ has at most $|F|$ variables and at most $|U|$ clauses. Now, by Theorem \ref{cnfsatmtime} there exists an algorithm that decides whether $\phi$ is satisfiable or not. So, we apply it and obtain the answer for the instance $(G,F)$.
\end{proof}

\subsection{Kernelization}

At first we introduce a definition of a crown decomposition proposed by Chor, et al. \cite{Crown}. Recall that for a disjoint sets $X,Y \subseteq V(G)$, a \textit{matching of $X$ into $Y$} is a matching $M$ such that every edge has one endpoint in $X$ and one endpoint in $Y$. In addition, for every $x \in X$ there exists exactly one edge $xy \in M$ such that $y \in Y$.

\begin{definition}
	A \textit{crown decomposition} of a graph $G$ is a partitioning of $V(G)$ into three parts $C,H$ and $R$, such that:
	\begin{itemize}
		\item $C$ is nonempty.
		\item $C$ is an independent set.
		\item There are no edges between $C$ and $H$. That is, $H$ separates $C$ from $R$.
		\item Let $E'$ be the set of edges between $C$ and $H$. Then, $E'$ contains a matching of size $|H|$. In other words, $G$ contains a matching $H$ into $C$.
	\end{itemize}
\end{definition}

Without digging into details, we show the results of Kőnig \cite{Konig}, Hall \cite{Hall} and an algorithm invented by Hopcroft and Karp \cite{Hopcroft-Karp} that are used further in this subsection.

\begin{theorem}[\textbf{Kőnig's theorem}]
	In every undirected bipartite graph the size of a maximum matching is equal to the size of a minimum vertex cover.
\end{theorem}

\begin{theorem}[\textbf{Hall's theorem}]
	Let $G$ be an undirected bipartite graph with bipartition $(V_1,V_2)$. The graph $G$ has a matching of $V_1$ into $V_2$ if and only if for all $X \subseteq V_1$, we have $|N(X)| \geq |X|$.
\end{theorem}

\begin{theorem}[\textbf{Hopcroft-Karp algorithm}]
	Let $G$ be an undirected Let G be an undirected bipartite graph with bipartition $(V_1,V_2)$, on $n$ vertices and $m$ edges.	Then, we can find a maximum matching as well as a minimum vertex cover of $G$ in time $\mathcal{O}(m\sqrt{n})$. Furthermore, in time $\mathcal{O}(m\sqrt{n})$ either we can find a matching of $V_1$ into $V_2$ or an inclusion-wise minimal set $X \subseteq V_1$ such that $|N(X)| < |X|$.
\end{theorem}

Now, we present the lemma proposed by Cygan, et al. \cite{ParAlg} %TODO%
that is a fundamental concept for designing kernelization algorithms.

\begin{lemma}
	Let $G$ be a graph without isolated vertices with at least $3k+1$ vertices. There is a polynomial-time algorithm that either
	\begin{itemize}
		\item finds a matching of size $k+1$ in $G$; or 
		\item finds a crown decomposition of $G$.
	\end{itemize} 
\end{lemma}

Finally, equipped with the above information, we are ready to show a linear kernel for \ssfep{}  parameterized by the number of free vertices.

\begin{lemma}
	There exists a polynomial-time algorithm, that, given an instance $(G,F)$ with a parameter $|U|$, either
	\begin{itemize}
		\item answers whether $(G,F)$ has a spanning star forest or
		\item outputs a subgraph of $(G,F)$ such that it has at most $|U|$ forced edges.
	\end{itemize} 
\end{lemma}


\begin{proof}
	Firstly, observe a trivial case. If $|F| \leq |U|$, then $(G,F)$ is the reduced instance. Otherwise, we apply Lemma \ref{norm-ssfe}. We either conclude that $(G,F)$ is a NO-instance, which satisfies the first bullet of the stated lemma, or obtain a normalized graph $(G',F')$. Observe that an isolated edge is a star and it cannot be expanded. Thus, we can remove them as well. In further, we consider only the case where $|F'| \geq |U'|$. If $|F'| < |U'|$, then $(G',F')$ is the reduced instance.
	
	Let $G_{F',U'}$ be a bipartite graph formed from $(G',F')$ by contracting forced edges. We apply Hopcroft-Karp's algorithm to $G_{F',U'}$. We either obtain a matching $M$ of $F'$ into $U'$ or an inclusion-wise minimal set $X \subseteq F'$ such that $|N(X)| < |X|$.
	\\\\
	Consider the first case in which Hoprocft-Karp's algorithm returns a matching. Then, we conclude that $(G,F)$ is a YES-instance due to the following claim:
	\begin{claim}
		Assume $(G,F)$ is a normalized instance of \ssfep{} where $|F| \leq |U|$. If Hoprocft-Karp's algorithm ran on $G_{F,U}$ returns a matching $M$ of $F$ into $U$, then $(G,F)$ is a YES-instance.
	\end{claim}

	\begin{proof}
		Since $M$ is a matching of $F$ into $U$, then we infer that $|F| \geq |U|$. Furthermore, we assumed that $|F| \leq |U|$. Thus, $|F|=|U|$ and $M$ is also a matching of $U$ into $F$. Now, we claim that there exists a spanning star forest for $(G,F)$. Indeed, let $M' \subseteq E(G')$ be a set of edges of minimum size such that $M'$ corresponds to $M$. Then, $M' \cup F$ is a spanning star forest for $G$.
	\end{proof}
	\noindent
	Observe that the normalized instance $(G',F')$ satisfies the assumptions from the above claim.
	\\\\
	Now, consider the second case. Let $X \subseteq$ be an inclusion-wise minimal set such that $|N(X)| < |X|$. Let $C=X$, $H=N(X)$ and $R= V(G_{F',U'}) \setminus (H \cup C)$. Now, we claim that:
	\begin{claim}
		$(C,H,R)$ is a crown decomposition of $G_{F',U'}$.
	\end{claim}
	\begin{proof}
		$C$ is nonempty and $C$ is an independent set as $F'$ is an induced matching in $G'$. Moreover, $H$ separates $C$ from $R$ because $H$ is a set of all the neighbors of $C$. Now, we prove that there exists a matching of $H$ into $C$. Select an arbitrary $v \in C$. There is a matching of $C \setminus \{v\}$ into $H$ since $|N(C')| \geq |C|$ for every $C' \subseteq C$. Since $|C| > |H|$, we have that the matching of $C \setminus \{v\}$ into $H$ is actually a matching of $H$ into $C$. Therefore, $(C,H,R)$ is a crown decomposition of $G_{F',U'}$.
	\end{proof}
	
\end{proof}

%\thmssfepkernel*


\section{Parametrization by treewidth}

\subsection{Preliminaries}

For \ssfep{}, we extend the notation of tree decomposition. Namely, for introduce vertex node, we distinguish \textit1{introduce free vertex node} and \textit{introduce forced vertex node}. We also extend the definition for introduce edge and forget vertex node consequently. Every cell of a dynamic table $\text{dp}$ has three parameters: a tree decomposition node $t$ and two assignment functions $f,g$. An \textit{assignment function for forced vertices} $f: (X_t \cap V(F)) \rightarrow \{\true, \false\}$ is a mapping that distinguishes two states. If $f(v)=1$, then we say that $v$ is a center whereas, if $f(v)=0$, then $v$ is either a candidate or a ray. An \textit{assignment function for free vertices} $g: (X_t \cap U) \rightarrow \{\true, \false\}$ is a mapping that indicates whether a free vertex is added to a star or not. Hence, for a free vertex $v$ we say that $v$ is in a star if $f(v)=1$ and, if it is not, then $f(v)=0$.

For the sake of clarity, we introduce the following notations. For an assignment function $f$ of $X$ and $v \in X$, we use $f_{|v}$ to denote the restriction of $f$ to $X \setminus \{v\}$. For a subset $X \subseteq V(G)$ consider an assignment function $f:X \rightarrow \{\true,\false\}$. For a vertex $v \in V(G)$ and a logic value $p \in \{\true, \false\}$ we define a new assignment $f_{v \rightarrow p}: X \cup \{v\} \rightarrow \{\true, \false\}$ as follows:

\begin{equation*}
	f_{v \rightarrow p} =
	\begin{cases}
	\begin{aligned}
		&f(u), & \text{if $u \neq v$} \\
		&p, &\text{if $u = v$}
	\end{aligned}
	\end{cases}
\end{equation*}

\subsection{Algorithm}

We provide formulas for every type of a node. To call a cell from a dynamic table we provide three arguments. The first one is a node $t$ from a tree decomposition. The second one is a forced vertices assignment $f:(X_t \cap V(F)) \rightarrow \{\true, \false\}$. The last one is  a free vertices assignment $g:(X_t \cap U) \rightarrow \{\true, \false\}$.

\paragraph{Leaf node} For a leaf node $t$ we have that $X_t=\emptyset$. An empty graph is a correct spanning star forest. Hence:

\begin{equation*}
	\dpt{t,\emptyset,\emptyset}=\true
\end{equation*}

\paragraph{Introduce forced vertex node} Let $t$ be an introduce node with a child $t'$ such that $X_t = X_{t'} \cup \{v\}$ and $v \in V(F)$. We simply assign an arbitrary value to the new vertex. Observe that we allow the case where for $vu \in F$, $f(v)=f(u)=\true$. However, we set the value for such a function to $\false$ during the introduce forced edge node. Thus, we get:

\begin{equation*}
	\dpt{t,f_{v \rightarrow p},g}= \dpt{t',f,g}\text{, for $p \in \{\false,\true\}$}
\end{equation*}

\paragraph{Introduce free vertex node} Let $t$ be an introduce free vertex node with a child $t'$ such that $X_t = X_{t'} \cup \{v\}$. Notice that the vertex $v$ is isolated in $G_t$. Therefore, we have the following formulas:

\begin{equation*}
	\dpt{t,f,g_{v \rightarrow p}} =
	\begin{cases}
		\dpt{t',f,g}, & \text{if $p=\false$} \\
		\false, &\text{otherwise}
	\end{cases}
\end{equation*}

\paragraph{Introduce free edge node} Let $t$ be an introduce free edge node labeled with an edge $vu \in E(G) \setminus F$ and let $t'$ be the child of $t$. Without loss of generality, assume that $v \in U$ and $u \in V(F)$ as every free edge has one end in a free vertex and the other one in a forced vertex. Let $f$ be a forced vertices assignment function and $g$ be a free vertices assignment function. Suppose that $g(v)=\false$. Then, we simply pass the value from the child's node. Otherwise, if $g(v)=\true$, we distinguish two cases. Firstly, let $f(u)=\false$. It means that $v$ was added to a star by another free edge. If $f(u)=\true$, then $v$ could be also added to a star by the edge $vu$. Thus, we obtain the following equation: 

\begin{equation*}
\begin{split}
	\dpt{t,f,g_{v \rightarrow \true}} & = 
		\begin{cases}
			\dpt{t',f,g_{v \rightarrow \true}} \lor \dpt{t',f,g_{v \rightarrow \false}}, & \text{if $f(u)$} \\
			\dpt{t',f,g_{v \rightarrow \true}}, & \text{otherwise}
		\end{cases}	
	\\
	\dpt{t,f,g_{v \rightarrow \false}} & = \dpt{t',f,g_{v \rightarrow \false}}\end{split}
\end{equation*}

\paragraph{Introduce forced edge node} Let $t$ be an introduce free edge node labeled with an edge $vu \in F$ and let $t'$ be the child of $t$ and let $f$ be a forced vertices assignment. Calculations for $t$ are simple. We set to $\false$ all the elements of dynamic table for which $f(v)=f(u)=\true$:

\begin{equation*}
	\dpt{t,f,g} =
	\begin{cases}
		\dpt{t',f,g}, & \text{if $\neg(f(v) \land f(u))$} \\
		\false, & \text{otherwise}
	\end{cases}
\end{equation*}

\paragraph{Forget forced vertex node} Let $t$ be an forget forced vertex node with a child $t'$, such that $X_t = X_{t'} \setminus \{v\}$ and let $u \in V(F)$ such that $vu \in F$. Observe that for any forced assignment function $f$, that satisfies $f(v)=f(u)=\true$, $\dpt{t',f} = \false$ because we have already changed their values during introduce forced edge node. Thus, the formula looks as follows:

\begin{equation*}
	\dpt{t,f_{|v},g} = \dpt{t',f,g}
\end{equation*}

\paragraph{Forget free vertex node} Let $t$ be an forget free vertex node with a child $t'$ such that $X_t = X_{t'} \setminus \{v\}$ where $v \in U$. We can pass the value from a child node if and only if $v$ was added to a star, that is, for a free vertices assignment function $g$, $g(v)=\true$. Consequently, we obtain:

\begin{equation*}
	\dpt{t,f,g_{|v}} =
	\begin{cases}
		\dpt{t',f,g}, & \text{if $g(v)$} \\
		\false, & \text{otherwise}
	\end{cases}
\end{equation*}

\paragraph{Join node} Let $t$ be a join node with children $t_1$ and $t_2$. Recall that $X_t=X_{t_1}=X_{t_2}$. We say that assignment functions $f_1,g_1$ of $X_{t_1}$ and $f_2,g_2$ of $X_{t_2}$ \textit{match} with assignments $f,g$ of $X_t$ if the following conditions hold:

\begin{enumerate}
	\item $f=f_1=f_2$, for forced vertices assignment functions $f,f_1,f_2$.
	\item $g(v)=g_1(v) \lor g_2(u)$, for every free vertex $v \in X_t \cap U$.
\end{enumerate}
Intuitively, we make sure that the centers remains at the same position and at least one of the children's assignments added every free vertex to a star. Therefore, we get the following equations:

\begin{equation*}
	\dpt{t,f,g} =
		\bigvee\limits_{g_1,g_2 \text{ match $g$}} dp[t_1,f_1,g_1] \land dp[t_2,f_2,g_2]
\end{equation*}

\subsection{Complexity analysis}

Observe that, except for a join node, every operation can be done in constant time. A naive approach to solve the disjunction would result in time $\mathcal{O}^*(2^{\tw(G)})$. Thus, we could conclude with an algorithm solving \ssfep{} parameterized by treewidth in time $\mathcal{O}^*(4^{\tw(G)})$ as there are $O(|G| \cdot 2^{\tw(G)})$ array entries. However, we can improve the run time. Thus, we want to introduce another approach. Consider the following definition:

\begin{definition}
	The \textit{cover product} of two functions $f,g:2^V \rightarrow \mathbb{Z}$ is a function $(f *_c g):2^V \rightarrow \mathbb{Z}$ such that for every $Y \subseteq V$:
	
	\begin{equation*}
		(f *_c g)(Y) = \sum\limits_{ A \cup B = Y} f(A)g(B)
	\end{equation*}
\end{definition}

Now, we state the theorem proved by Björklund, Husfeldt, Kaski and Koivisto \cite{Cover product}:

\begin{theorem}\label{cproduct}
	For two functions $f,g:2^V \rightarrow \mathbb{Z}$, given all $2^n$ values of $f$ and $g$ in the input, all the $2^n$ values of the cover product $f*_cg$ can be computed in $\mathcal{O}(2^n\cdot n)$ arithmetic operations.
\end{theorem}

Notice that the disjunction in every join node is nothing different than a cover product for a fixed forced vertices assignment. Thus, we formulate the lemma:

\begin{lemma}\label{join lemma}
	Given a join node $t$, one can calculate all its values in time $O(2^{\tw(G)})$.
\end{lemma}

\begin{proof}
	Fix a forced vertices assignment $f$. We define a function $c_{t,f}:2^{X_t \cap U} \rightarrow \mathbb{Z}$ as follows:
	
	\begin{equation*}
		c_{t,f}(X) = dp[t,f,g] \text{, such that $g^{-1}(1) = X$}
	\end{equation*}	
	Now, for $X \subseteq X_t \cap U$, observe that: 
	\begin{align*}
		(c_{t_1,f} *_c c_{t_2,f})(X) &= \sum\limits_{A \cup B = X} c_{t_1,f}(A)c_{t_2,f}(B) \\
		&= \sum\limits_{ g_1^{-1}(1)\ \cup\ g_2^{-1}(1) = X} dp[t_1,f,g_1]dp[t_2,f,g_2]	
	\end{align*}
	which exactly reflects the calculation that we do during a join node. Thus, $\dpt{t,f,g} = (c_{t_1,f} *_c c_{t_2,f})(g^{-1}(1)) > 0$. 
	
	There are $2^{|X_t \cap V(F)|}$ different forced vertices assignments. For a given forced vertices assignment $f$, by Theorem \ref{cproduct}, we can calculate values for $f$ and every possible free vertices assignment $g$ in time $2^{|X_t \cap U|}$. Thus, for a join node $t$ we can fill its dynamic table cells in time $2^{|X_t \cap V(F)|} \cdot 2^{|X_t \cap U|} = 2^{|X_t|} \leq 2^{\tw(G)}$.
\end{proof}

\thmssfeptwtime*

\begin{proof}
	Consider the algorithm described in the previous subsection. To calculate a single entry for introduce and forget nodes, we need constant time. By Lemma \ref{join lemma}, we showed that that the values for a join node can be calculated in $\mathcal{O}(2^{\tw(G)})$. There are polynomially many nodes in a tree decomposition. Thus, we can fill the values of a dynamic table in time $\mathcal{O}^*(2^{\tw(G)})$ and provide the answer whether the input graph has a \ssf{}.
\end{proof}

\thmssfeptwseth*

\begin{proof}
	Let $\phi$ be an arbitrary instance of \cnfsat{} problem with $n$ variables. We apply a reduction from Lemma \ref{ssfep reduction} and obtain an equivalent instance $(G,F)$. Now, we need to prove that $tw(G) \leq n$. Thus we propose the following path decomposition. Let $B_1=\{v[x_i]: 1 \leq i \leq n\}$. Then, we iteratively introduce and forget every vertex $v \in N[B_1] \cap U$. Next, for every vertex $v[x_i]$ we introduce vertex $v[\neg x_i]$ and forget $v[x_i]$. Finally, we repeat the second step, that is, we iteratively introduce and forget every vertex $v \in N[\{v[\neg x_i]: 1 \leq i \leq n\}] \cap U$. Observe that every bag of the decomposition consists of at most $n+1$ vertices. Thus, $pw(G)=n$. Since $tw(G) \leq pw(G)$ we conclude that $tw(G) \leq n$.
	
	To conclude, observe that if there was an algorithm solving \ssfep{} parameterized by treewidth in $2^{o(
		 \tw(G))}$, then it would contradict SETH.
\end{proof}

\begin{thebibliography}{99}
\addcontentsline{toc}{chapter}{Bibliografia}

\bibitem{ParComp}  Downey, R., Fellows, M.: \textit{Parameterized Complexity}. Springer-Verlag, New York (1999).

\bibitem{ParCompThm} Flum, J., Grohe, M.: \textit{Parameterized Complexity Theory}. Texts in Theoretical Computer Science. An EATCS Series. Springer-Verlag, Berlin (2006).

\bibitem{ParAlg} Cygan, M., Fomin, F., Kowalik L., Lokshtanov, D., Marx, D., Pilipczuk, M., Pilipczuk, M., Saurabh, S.: \textit{Parameterized Algorithms}. Springer, (2015).

\bibitem{DomSet} Pătraşcu, M., Williams, R.: \textit{On the Possibility of Faster SAT Algorithms}. Proceedings of the 20th Annual ACM-SIAM Symposium on Discrete Algorithms (SODA), pp. 1065-1075. SIAM (2010).

\bibitem{DomSetAlg} Eisenbrand F., Grandoni F.: \textit{On the
Complexity of Fixed Parameter Clique and Dominating Set}. Theor. Comput. Sci. 326, pp. 57–67, (2004).

\bibitem{Crown} Chor, B., Fellows, M, Juedes, D.: \textit{Linear kernels in linear time, or how to save k colors in $O(n^2)$ steps}. WG'04 Proceedings of the 30th international conference on Graph-Theoretic Concepts in Computer Science, pp. 257-269, (2004).

\bibitem{Konig} König, D.: \textit{Über Graphen und ihre Anwendung auf Determinantentheorie und Mengenlehre}. Math. Ann. 77(4), pp. 453-465, (1916)
\bibitem{Hall} Hall, P.: \textit{On representatives of subsets}, J. London Math. Soc. 10, pp. 26-30, (1935).

\bibitem{Hopcroft-Karp} Hopcroft, J.E., Karp, R.M.: \textit{An $n^{5/2}$ algorithm for maximum matchings in bipartite graphs}. SIAM J. Computing 2, pp. 225-231, (1973).

\bibitem{Bodlaender} Bodlaender, H., Downey, R., Fellows, M., Hermelin, D.: \textit{On problems without polynomial kernels}. Proceedings of 35th International Colloquium on Automata, Languages and Programming, pp. 563–574, (2008).

\bibitem{CNFSAT} Fortnow, L., Santhanam, R.: \textit{Infeasibility of instance compression and succinct pcps for NP}. Journal of Computer and System Sciences, 77(1), pp. 91–106, (2011).

\bibitem{Cover product} Björklund, A., Husfeldt, T., Kaski, P., Koivisto, M.: \textit{Fourier meets Möbius: fast subset convolution}. Proceedings of the 39th Annual ACM Symposium on Theory of Computing (STOC), pp. 67-74. ACM, New York (2007).



\end{thebibliography}

\end{document}


%%% Local Variables:
%%% mode: latex
%%% TeX-master: t
%%% coding: latin-2
%%% End:
